\section*{Приложение А}
\vspace{-5mm} % Уменьшает отступ
\section*{\lowercase{(обязательное)}}
\vspace{-5mm} % Уменьшает отступ
\section*{Л\lowercase{истинг программного обеспечения}}
\addcontentsline{toc}{section}{Приложение А (обязательное) Листинг программного обеспечения}



\begin{verbatim}
main.cpp
	
#include "voice_assistant.h"

int main() {
	Core::VoiceAssistant assistant;
	assistant.run();
	return 0;
}

voice_assistant.h

#pragma once
#include "command_manager.h"
#include "speech_recognizer.h"
#include <memory>

namespace Core {
	class VoiceAssistant {
		public:
		VoiceAssistant();
		void run();
		
		private:
		std::unique_ptr<Speech::SpeechRecognizer>
		 _recognizer;
		Commands::CommandManager _commandManager;
		
		void processCommand(const std::string& text);
	};
} // namespace Core

voice_assistant.cpp

#include "voice_assistant.h"

namespace Core {
	VoiceAssistant::VoiceAssistant() {
		_recognizer = std::make_unique<
		Speech::SpeechRecognizer>(
		"../../Resources/models/"
		"vosk-model-small-ru-0.22",
		"[\"алиса\", \"включить станок\",
		 \"выключить станок\", \"подать "
		"масло\"]");
	}
	
	void VoiceAssistant::processCommand(
	const std::string& text) {
		_commandManager.process(text);
	}
	
	void VoiceAssistant::run() {
		const char* logo = R"(
	\end{verbatim}
	{\tiny  % или \footnotesize, \scriptsize, \tiny
		\begin{verbatim}
			__     __    _               _            _     _              _   
			\ \   / /__ (_) ___ ___     / \   ___ ___(_)___| |_ __ _ _ __ | |_ 
			\ \ / / _ \| |/ __/ _ \   / _ \ / __/ __| / __| __/ _` | '_ \| __|
			\ V / (_) | | (_|  __/  / ___ \\__ \__ \ \__ \ || (_| | | | | |_ 
			\_/ \___/|_|\___\___| /_/   \_\___/___/_|___/\__\__,_|_| |_|\__|
		\end{verbatim}
	}
	\begin{verbatim}
		std::cout << logo;
		std::cout << "Голосовой помощник запущен\n";
		while (true) {
			auto text = _recognizer->listen();
			if (!text.empty()) {
				processCommand(text);
			}
		}
	}
	
} // namespace Core

speech_recognizer.h

#pragma once

#include <portaudio.h>
#include <vosk_api.h>

namespace Speech {
	class SpeechRecognizer {
		public:
		SpeechRecognizer(const std::string& modelPath,
		const std::string& grammarJson);
		~SpeechRecognizer();
		
		std::string listen();
		
		private:
		VoskModel* _model = nullptr;
		VoskRecognizer* _recognizer = nullptr;
		PaStream* _stream = nullptr;
		PaStreamParameters _inputParameters{};
	};
} // namespace Speech

speech_recognizer.cpp

#include "speech_recognizer.h"

namespace Speech {
	constexpr int sampleRate = 16000;
	constexpr int framesPerBuffer = 4000;
	
	SpeechRecognizer::SpeechRecognizer(
	const std::string& modelPath,
	const std::string& grammarJson) {
		_model = vosk_model_new(modelPath.c_str());
		_recognizer = vosk_recognizer_new_grm(_model,
		sampleRate,
		grammarJson.c_str());
		
		PaError err = Pa_Initialize();
		if (err != paNoError) {
			std::cerr << "Ошибка инициализации PortAudio: "
			<< Pa_GetErrorText(err) << "\n";
			exit(1);
		}
		
		_inputParameters.device =
		 Pa_GetDefaultInputDevice();
		if (_inputParameters.device == paNoDevice) {
			std::cerr <<
			 "Нет доступного входного устройства.\n";
			exit(1);
		}
		
		_inputParameters.channelCount = 1;
		_inputParameters.sampleFormat = paInt16;
		_inputParameters.suggestedLatency
		= Pa_GetDeviceInfo(
		_inputParameters.device)->defaultLowInputLatency;
		_inputParameters.hostApiSpecificStreamInfo
		 = nullptr;
		
		err = Pa_OpenStream(&_stream,
		&_inputParameters,
		nullptr,
		sampleRate,
		framesPerBuffer,
		paClipOff,
		nullptr,
		nullptr);
		if (err != paNoError) {
			std::cerr << "Ошибка открытия потока: "
			 << Pa_GetErrorText(err) << "\n";
			exit(1);
		}
		
		Pa_StartStream(_stream);
	}
	
	std::string SpeechRecognizer::listen() {
		short buffer[framesPerBuffer];
		Pa_ReadStream(_stream, buffer, framesPerBuffer);
		
		if (vosk_recognizer_accept_waveform(_recognizer,
		reinterpret_cast<const char*>(
		buffer),
		framesPerBuffer * sizeof(short))) {
			const char* result = vosk_recognizer_result(
			_recognizer);
			std::string text(result);
			
			auto pos1 = text.find("\"text\" : \"");
			if (pos1 != std::string::npos) {
				pos1 += 10;
				auto pos2 = text.find("\"", pos1);
				return text.substr(pos1, pos2 - pos1);
			}
		}
		return "";
	}
	
	SpeechRecognizer::~SpeechRecognizer() {
		Pa_StopStream(_stream);
		Pa_CloseStream(_stream);
		Pa_Terminate();
		
		vosk_recognizer_free(_recognizer);
		vosk_model_free(_model);
	}
} // namespace Speech

command_manager.h

#pragma once

#include <functional>
#include <unordered_map>

namespace Commands {
	class CommandManager {
		public:
		CommandManager();
		
		void process(const std::string& text);
		
		private:
		struct CommandData {
			std::string voiceTrigger;
			std::function<void()> handler;
		};
		
		std::unordered_map<std::string,
		 CommandData> _commands;
		
		template<typename F>
		void registerCommand(std::string name,
		std::string voiceTrigger,
		F&& handler) {
			_commands.emplace(std::move(name),
			CommandData{std::move(voiceTrigger),
				std::forward<F>(handler)});
		}
		
		void logTime();
	};
} // namespace Commands

command_manager.cpp

#include "command_manager.h"
#include <chrono>
#include <iomanip>

namespace Commands {
	CommandManager::CommandManager() {
		registerCommand("start_machine",
		 "включить станок", []() {
			std::cout << "Станок включен\n";
		});
		registerCommand("stop_machine",
		 "выключить станок", []() {
			std::cout << "Станок выключен\n";
		});
		registerCommand("oil_feed",
		 "подать масло", []() {
			std::cout << "Масло подано\n";
		});
	}
	
	void CommandManager::process(
	const std::string& text) {
		if (text.find("алиса") ==
		 std::string::npos) {
			return;
		}
		
		for (const auto& [name, data] : _commands) {
			if (text.find(data.voiceTrigger) !=
			 std::string::npos) {
				logTime();
				data.handler();
				return;
			}
		}
	}
	
	void CommandManager::logTime() {
		auto now = std::chrono::system_clock::now();
		auto now_time_t =
		 std::chrono::system_clock::to_time_t(now);
		std::cout << "["
		<< std::put_time(std::localtime(&now_time_t),
		"%Y-%m-%d %H:%M:%S")
		<< "] ";
	}
} // namespace Commands
\end{verbatim}

\newpage
