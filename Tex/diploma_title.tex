\begin{titlepage}
	\begin{center}
		Министерство образования Республики Беларусь\\[1em]
		Учреждение образования\\
		БЕЛОРУССКИЙ ГОСУДАРСТВЕННЫЙ УНИВЕРСИТЕТ \\
		ИНФОРМАТИКИ И РАДИОЭЛЕКТРОНИКИ\\[1em]
		
		\begin{minipage}{\textwidth}
			\begin{flushleft}
				\begin{tabular}{ l l }
					Факультет информационной безопасности \\
					Кафедра инфокоммуникационных технологий
				\end{tabular}
			\end{flushleft}
		\end{minipage}\\[1em]
		
		\begin{flushright}
			\begin{minipage}{0.4\textwidth}
				\textit{К защите допустить:}\\[0.8em]
				Заведующий кафедрой ИКТ\\[0.45em]
				\underline{\hspace*{2.8cm}} В.\,Ю.~Цветков
			\end{minipage}\\[2.2em]
		\end{flushright}
		
		%%
		%% ВНИМАНИЕ: на некторых факультетах (ФКП) и кафедрах (ПИКС) слова "ПОЯСНИТЕЛЬНАЯ ЗАПИСКА" предлагается (требуется) оформлять полужирным начертанием. Раскомментируйте нужную для вас строку:
		%%
		%\textbf{ПОЯСНИТЕЛЬНАЯ ЗАПИСКА}\\
		{ПОЯСНИТЕЛЬНАЯ ЗАПИСКА}\\
		{к дипломному проекту}\\
		{на тему:}\\[1em]
		\textbf{\large ГОЛОСОВОЙ ПОМОЩНИК ДЛЯ ОДНОПЛАТНОГО КОМПЬЮТЕРА}\\[1em]
		
		
		{БГУИР ДП 1-45 01 01 05 066 ПЗ}\\[2em]
		
		\begin{tabular}{ p{0.65\textwidth}p{0.25\textwidth} }
			Студент & И.\,И.~Иванов \\
			Руководитель & Иванов\,Иван \\
			Консультанты: &\\
			\hspace*{3ex}\emph{по экономической части} & И.\,И.~Иванов \\
			%%
			%% ВНИМАНИЕ: в зависимости от выбранной темы, у вас консультант может быть как по охране труда, так и по:
			% экологической безопасности
			% ресурсосбережению
			% энергосбережению
			%%
			%% Впишите правильную формулировку по необходимости
			Нормоконтролёр & И.\,И.~Иванов\\
			& \\
			Рецензент & И.\,И.~Иванов
		\end{tabular}
		
		\vfill
		{\normalsize Минск 2025}
	\end{center}
\end{titlepage}
