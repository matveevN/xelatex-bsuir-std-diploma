\section*{Перечень условных обозначений, символов и терминов}

\addcontentsline{toc}{section}{Перечень условных обозначений, символов и терминов}



% Список обозначений
\begin{description}
	\item[\normalfont \textit{ASR}] (\textit{Automatic Speech Recognition}) -- технология преобразует устную речь в текст с помощью алгоритмов машинного обучения и лингвистического анализа.
	
	\item[\normalfont \textit{CPU}] (\textit{Central Processing Unit}) -- центральный процессор, основное устройство для выполнения инструкций программ.
	
	\item[\normalfont \textit{CSI}] (\textit{Camera Serial Interface}) -- это интерфейс, который используется для передачи видеоданных от камеры к процессору по нескольким линиям данных и одной линии тактирования.
	
	\item[\normalfont \textit{DSI}] (\textit{Display Serial Interface}) -- это высокоскоростной серийный интерфейс для подключения дисплеев (особенно сенсорных экранов) к процессорам и микроконтроллерам.
	
	\item[\normalfont \textit{E2E}] (\textit{End-to-End}) -- это подход, при котором система преобразует входные аудиоданные напрямую в текст без промежуточных этапов таких как разделения на фонемы или словари.
	
	\item[\normalfont \textit{GPIO}] (\textit{General Purpose Input/Output}) -- это универсальные пины, которые можно программно настраивать как входы или выходы для управления или считывания сигналов.
	
	\item[\normalfont \textit{GPU}] (\textit{Graphics Processing Unit}) -- графический процессор.
	
	\item[\normalfont \textit{MicroSD}] (\textit{Micro Secure Digital Card}) -- карта памяти малого формата.
	
	\item[\normalfont \textit{NLP}] (\textit{Natural Language Processing}) -- это область искусственного интеллекта, которая занимается анализом, пониманием и генерацией человеческого языка, текста и речи.
	
	\item[\normalfont \textit{PMIC}] (\textit{Power Management Integrated Circuit}) -- это микросхема, предназначенная для управления питанием в электронных устройствах.
	
	\item[\normalfont \textit{RAM}] (\textit{Random Access Memory}) -- оперативная память, используемая для временного хранения данных и инструкций, с которыми работает процессор.
	
	\item[\normalfont \textit{SoC}] (\textit{System-on-Chip}) -- это интегральная схема, которая объединяет все ключевые компоненты электронного устройства на одном кристалле кремния.
	
	\item[\normalfont \textit{USB}] (\textit{Universal Serial Bus}) -- это стандарт для подключения, передачи данных и питания между компьютерами и периферийными устройствами.
\end{description}


\newpage
