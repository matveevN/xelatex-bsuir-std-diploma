\section*{Перечень условных обозначений, символов и терминов}

\addcontentsline{toc}{section}{Перечень условных обозначений, символов и терминов}



% Список обозначений
\begin{description}
	\item[\normalfont ASP] (Automatic Speech Recognition) -- технология преобразует устную речь в текст с помощью алгоритмов машинного обучения и лингвистического анализа.
	
	\item[\normalfont CPU] (Central Processing Unit) -- центральный процессор.
	
	\item[\normalfont CSI] (Camera Serial Interface) -- это интерфейс, который используется для передачи видеоданных от камеры к процессору по нескольким линиям данных и одной линии тактирования.
	
	\item[\normalfont DSI] (Display Serial Interface) -- это высокоскоростной серийный интерфейс для подключения дисплеев (особенно сенсорных экранов) к процессорам и микроконтроллерам.
	
	\item[\normalfont E2E] (End-to-End) -- это подход, при котором система преобразует входные аудиоданные напрямую в текст без промежуточных этапов таких как разделения на фонемы или словари.
	
	\item[\normalfont GPIO] (General Purpose Input/Output) -- это универсальные пины, которые можно программно настраивать как входы или выходы для управления или считывания сигналов.
	
	\item[\normalfont GPU] (Graphics Processing Unit) -- графический процессор.
	
	\item[\normalfont MicroSD] (Micro Secure Digital Card) -- .
	
	\item[\normalfont NLP] (Natural Language Processing) -- это область искусственного интеллекта, которая занимается анализом, пониманием и генерацией человеческого языка, текста и речи.
	
	\item[\normalfont PMIC] (Power Management Integrated Circuit) -- это микросхема, предназначенная для управления питанием в электронных устройствах.
	
	\item[\normalfont RAM] (Random Access Memory) -- оперативная память.
	
	\item[\normalfont SoC] (System-on-Chip) -- это интегральная схема, которая объединяет все ключевые компоненты электронного устройства на одном кристалле кремния.
	
	\item[\normalfont USB] (Universal Serial Bus) -- это стандарт для подключения, передачи данных и питания между компьютерами и периферийными устройствами.
	

\end{description}

\newpage
