\section*{Реферат}
\thispagestyle{empty} % Снятие нумерации с текущей страницы

\noindent БГУИР ДП 1-45 01 01 066 ПЗ
\newline


\nohyphens{ГОЛОСОВОЙ ПОМОЩНИК ДЛЯ ОДНОПЛАТНОГО КОМПЬЮТЕРА:} дипломный проект / Н. В. Матвеев. -- Минск : БГУИР, 2025, -- п.з. -- \pageref{LastPage}~с., чертежей -- 3 л. формата А1, плакатов -- 3 л. формата А1.
\newline

ГОЛОСОВОЙ ПОМОЩНИК, РАСПОЗНАВАНИЕ РЕЧИ, АВТОНОМНАЯ СИСТЕМА, ОДНОПЛАТНЫЙ КОМПЬЮТЕР RASPBERRY PI 4 MODEL B, БИБЛИОТЕКА VOSK, PORTAUDIO, C++, МОДУЛЬНАЯ АРХИТЕКТУРА, УПРАВЛЕНИЕ ГОЛОСОМ, ОФЛАЙН-РАСПОЗНАВАНИЕ, ГОЛОСОВАЯ АКТИВАЦИЯ. 
\newline

Дипломный проект посвящен разработке голосового помощника для одноплатного компьютера Raspberry Pi 4 Model B. Основной целью проекта является создание программного обеспечения, позволяющего управлять устройством с помощью голосовых команд в автономном режиме, без необходимости подключения к интернету.

В проекте рассмотрены методы обработки естественного языка, алгоритмы распознавания речи и интеграция с аппаратной платформой. Для реализации использованы технологии Vosk и PortAudio, а также язык программирования C++. Проведено тестирование системы, оценены точность распознавания команд и быстродействие. Результаты показали, что голосовой помощник корректно функционирует даже в условиях фонового шума, потребляя при этом не более 20\% ресурсов процессора и 300 Мб оперативной памяти.

Разработанное решение может быть применено в системах «умного дома», промышленной автоматизации и на образовательных платформах.

\clearpage
