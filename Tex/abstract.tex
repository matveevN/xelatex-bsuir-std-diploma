\section*{Реферат}
\thispagestyle{empty} % Снятие нумерации с текущей страницы

\nohyphens{ГОЛОСОВОЙ ПОМОЩНИК ДЛЯ ОДНОПЛАТНОГО КОМПЬЮТЕРА}: дипломный проект / Н.В. Матвеев. – Минск : БГУИР, 2025, - п.з. – \pageref{LastPage}~с., чертежей (плакатов) – 6 л. формата А1.

Дипломный проект посвящён разработке голосового помощника для одноплатного компьютера Raspberry Pi 4 Model B. Основной целью работы является создание программного обеспечения, позволяющего управлять устройством с помощью голосовых команд в автономном режиме, без необходимости подключения к интернету.

В проекте рассмотрены методы обработки естественного языка, алгоритмы распознавания речи и интеграция с аппаратной платформой. Для реализации использованы технологии Vosk и PortAudio, а также язык программирования C++. Проведено тестирование системы, оценены точность распознавания команд и быстродействие. Результаты показали, что голосовой помощник корректно функционирует даже в условиях фонового шума, потребляя при этом не более 20\% ресурсов процессора и 300 МБ оперативной памяти.

Разработанное решение может быть применено в системах «умного дома», промышленной автоматизации и образовательных проектах. Проект включает программную реализацию и техническую документацию.

\clearpage
