\section{Проектирование и разработка программного средства}

\subsection{Архитектура программного средства}

Процесс разработки программного обеспечения для голосового помощника для одноплатного компьютера требует внимательного подхода к проектированию, с учетом особенностей работы на ограниченных ресурсах и необходимости обеспечения масштабируемости. Как показано на рисунке~\ref{fig:class_diagram}, для решения этих задач был выбран объектно-ориентированный подход, обеспечивающий четкое разделение функциональности на независимые компоненты.

Для наглядного описания архитектуры системы и взаимосвязей между её компонентами, в данном проекте используется унифицированный язык моделирования UML. UML является стандартным инструментом для визуализации, специфицирования, конструирования и документирования артефактов программных систем, что делает его особенно полезным при разработке сложных программных решений~\cite{UML}.

\begin{figure}[H]
	\centering
	\fbox{\includegraphics[scale=0.26]{class_Diagram.jpg}}
	\caption{Диаграмма классов}
	\label{fig:class_diagram}
\end{figure}
На рисунке~\ref{fig:class_diagram} показано что программное средство состоит из следующих основных компонентов:

\begin{itemize}
	\item {Core::VoiceAssistant} -- центральный компонент, управляющий жизненным циклом голосового помощника. Он отвечает за инициализацию всех необходимых модулей и запуск основного цикла обработки команд.
	
	\item {Speech::SpeechRecognizer} -- модуль распознавания речи, использующий библиотеку Vosk и PortAudio для захвата и анализа аудиопотока с микрофона.
	
	\item {Commands::CommandManager} -- отвечает за регистрацию голосовых команд и выполнение соответствующих обработчиков при их распознавании.
\end{itemize}

Взаимодействие между компонентами организовано следующим образом: VoiceAssistant инициализирует экземпляр SpeechRecognizer, а затем в бесконечном цикле получает голосовой ввод, преобразованный в текст. Полученный текст передается в CommandManager, который анализирует его и при обнаружении соответствующей команды вызывает заранее зарегистрированную функцию-обработчик.

Такой подход позволяет легко расширять систему, добавляя новые голосовые команды, не затрагивая существующую логику работы помощника.

Представленная архитектура с модульным разделением компонентов позволяет системе поддерживать различные сценарии взаимодействия. На рисунке \ref{fig:use_case_diagram} показана диаграмма вариантов использования, иллюстрирующая основные функциональные возможности голосового помощника и взаимодействие с пользователями разных категорий.

\begin{figure}[H]
	\centering
	\fbox{\includegraphics[scale=0.24]{use_Case_Diagram.jpg}}
	\caption{Диаграмма вариантов использования}
	\label{fig:use_case_diagram}
\end{figure}

Диаграмма вариантов использования иллюстрирует ключевые функциональные возможности системы, разделенные по категориям пользователей. Основной функционал доступен всем пользователям и включает базовые операции: запуск/остановку системы и управление подключенным оборудованием. Эти функции представляют собой ядро системы, необходимое для повседневной работы с голосовым помощником.

Для администраторов предусмотрен расширенный набор возможностей, включающий удаленное подключение к системе и просмотр системных логов. Эти функции предназначены для технического обслуживания, мониторинга работы системы и устранения неполадок.

Система четко разделяет пользователей на две категории:
\begin{itemize}
	\item обычные пользователи -- работают только с базовым функционалом;
	\item администраторы -- обладают полным доступом ко всем возможностям системы.
\end{itemize}

Такая архитектура прав доступа обеспечивает оптимальный баланс между удобством использования и безопасностью. Обычные пользователи получают простой и интуитивно понятный интерфейс, в то время как администраторы имеют все необходимые инструменты для управления системой.


Представленный алгоритм работы голосового помощника демонстрирует последовательность действий системы при обработке пользовательских запросов. На рисунке \ref{fig:algorithm} изображена блок-схема, наглядно иллюстрирующая ключевые этапы взаимодействия пользователя с системой.

\begin{figure}[H]
	\centering
	\fbox{\includegraphics[scale=0.8]{algorithm.jpg}}
	\caption{Алгоритм работы голосового помощника}
	\label{fig:algorithm}
\end{figure}

Алгоритм начинается с инициализации распознавания речи, после чего система переходит к выбору логотипа и анализу считанных и распознанных данных. Важным этапом является проверка наличия ключевого слова "Алиса", которое служит триггером для дальнейших действий. Если ключевое слово обнаружено, система выполняет поиск соответствующей команды. В случае успешного нахождения команды, она выполняется, а результат выводится на экран. Процесс завершается при нажатии комбинации клавиш Ctrl + C.

Такой алгоритм обеспечивает четкую и предсказуемую работу голосового помощника, позволяя пользователю эффективно взаимодействовать с системой. Разделение на этапы способствует упрощению отладки и дальнейшего развития функциональности.

Архитектура системы голосового помощника представляет собой модульную структуру, обеспечивающую эффективное взаимодействие между пользователем и техническими компонентами. На рисунке \ref{fig:system_architecture_diagram} показана схема, детализирующая ключевые элементы системы и их взаимосвязи.

\begin{figure}[H]
	\centering
	\fbox{\includegraphics[scale=1]{system_architecture_diagram.jpg}}
	\caption{Схема архитектуры взаимодействия оператор-машина}
	\label{fig:system_architecture_diagram}
\end{figure}

Система начинается с эмулятора терминала, который служит интерфейсом для ввода команд через Bash-скрипты. Пользователь может взаимодействовать с системой как через текстовый ввод, так и через голосовой ввод, который преобразуется в текст с помощью модуля распознавания речи. Далее обработчик команд анализирует полученные данные и определяет соответствующие действия.

Основной функционал системы реализуется через голосового помощника, который координирует работу всех компонентов. Для управления внешними устройствами используется специализированный модуль управления устройствами, а блок исполнения команд обеспечивает выполнение запрошенных операций.

Такая архитектура обеспечивает:
\begin{itemize}
	\item гибкость за счет модульного разделения компонентов;
	\item поддержку многомодального ввода (текст/голос);
	\item централизованное управление устройствами;
	\item масштабируемость для добавления новых функций.
\end{itemize}

Структурная схема автоматизированного рабочего места инженера демонстрирует аппаратную конфигурацию системы и взаимосвязи между компонентами. На рисунке \ref{fig:connection_scheme} представлена компоновка оборудования, обеспечивающего полноценную работу системы голосового управления.
\begin{figure}[H]
	\centering
	\fbox{\includegraphics[scale=0.8]{connection_sheme.jpg}}
	\caption{Структурная схема автоматизированного рабочего места инженера}
	\label{fig:connection_scheme}
\end{figure}

На представленной схеме изображена конфигурация автоматизированного рабочего места инженера, построенная на базе одноплатного компьютера, который выполняет роль центрального вычислительного узла системы. К компьютеру подключены основные периферийные устройства, обеспечивающие полноценное функционирование рабочей станции.

В качестве устройств ввода информации используются стандартная клавиатура и манипулятор типа "мышь", позволяющие осуществлять ручное управление системой. Для визуального отображения информации предназначен монитор, являющийся основным устройством вывода. Голосовой вывод реализован через наушники, что особенно важно для систем с голосовым управлением.

Энергоснабжение всей системы обеспечивает блок питания (БП), рассчитанный на питание всех компонентов рабочей станции. В качестве основного носителя данных и программного обеспечения используется SD-карта, установленная в одноплатный компьютер. Для обеспечения совместимости и соединения всех компонентов между собой применяются различные переходники и интерфейсные преобразователи.

Данная конфигурация обладает рядом преимущественных особенностей. Использование одноплатного компьютера обеспечивает компактность и мобильность всего рабочего места. При этом система сохраняет универсальность за счет возможности подключения стандартных периферийных устройств. Энергоэффективная архитектура решения позволяет минимизировать энергопотребление, сохраняя при этом необходимую вычислительную мощность.

Такая схема организации рабочего места представляет собой оптимальное решение для инженерных задач, сочетающее в себе современные технологии автоматизации, надежность работы и удобство эксплуатации. Особенно эффективно данное решение проявляет себя в системах, требующих поддержки голосового управления и мобильности рабочей станции.

\subsection{Реализация основных модулей}

На представленном рисунке \ref{fig:main.cpp} показан исходный код на языке C++, реализующий точку входа в систему голосового помощника.
\begin{figure}[H]
	\centering
	\includegraphics[scale=0.7]{main.jpg}
	\caption{Точка входа в программную систему}
	\label{fig:main.cpp}
\end{figure}

При создании объекта assistant вызывается конструктор класса VoiceAssistant на рисунке \ref{fig:constructor_VoiceAssistant}

\begin{figure}[H]
	\centering
	\includegraphics[scale=0.8]{constructor_VoiceAssistant.jpg}
	\caption{Конструктор VoiceAssistant}
	\label{fig:constructor_VoiceAssistant}
\end{figure}

Конструктор класса VoiceAssistant:
\begin{itemize}
 	\item загружается модель распознавания речи Vosk для русского языка;
	\item настраивается ограниченный набор фраз, которые ассистент сможет распознавать.
\end{itemize}

Это повышает точность, так как система не пытается распознать произвольную речь, а ждет только указанные команды.

Затем созданный объекта вызывает метод run() который представлен на рисунке \ref{fig:run_VoiceAssistant}.

\begin{figure}[H]
	\centering
	\includegraphics[scale=0.9]{run_VoiceAssistant.jpg}
	\caption{Метод run() класса VoiceAssistant}
	\label{fig:run_VoiceAssistant}
\end{figure}

Метод VoiceAssistant::run() обеспечивает запуск и непрерывную работу голосового ассистента. В начале работы выводится ASCII-графика в виде стилизованного логотипа. Сразу после отображения логотипа программа выводит текстовое сообщение "Голосовой помощник запущен". Основная функциональность реализована через бесконечный цикл, в котором есть объект который создан через конструктор показанный на рисунке \ref{fig:constructor_SpeechRecognizer}.

\begin{figure}[H]
	\centering
	\includegraphics[scale=0.5]{constructor_SpeechRecognizer.jpg}
	\caption{Конструктор SpeechRecognizer}
		\label{fig:constructor_SpeechRecognizer}
\end{figure}
и в котором непрерывно происходит захват аудиопотока через метод listen() представленном на рисунке \ref{fig:listenSpeechRecognizer}.

\begin{figure}[H]
	\centering
	\includegraphics[scale=0.8]{listenSpeechRecognizer.jpg}
	\caption{Метод listen() класса SpeechRecognizer}
	\label{fig:listenSpeechRecognizer}
\end{figure}

Данные метод реализует процесс захвата и обработки аудиоданных для распознавания речи. В начале работы создается буфер размером framesPerBuffer для хранения аудиосэмплов в 16-битном формате. С помощью функции Pa\_ReadStream осуществляется чтение аудиопотока из микрофона, при этом данные записываются в подготовленный буфер. Полученные аудиоданные передаются в распознаватель Vosk через функцию vosk\_recognizer\_accept\_waveform, где происходит их преобразование в текстовый формат.

Если распознавание прошло успешно, функция vosk\_recognizer\_result возвращает результат в виде JSON-строки, содержащей распознанный текст. Из этой строки извлекается непосредственно текстовая часть между полями \string"text\string" : \string" и закрывающей кавычкой. Для этого используется поиск позиций соответствующих подстрок и последующее выделение нужного фрагмента. Если текст успешно извлечен, он возвращается как результат работы метода. В случае неудачного распознавания или отсутствия текста в результате возвращается пустая строка.

Таким образом, метод обеспечивает непрерывный цикл захвата аудио, его преобразование в текст и извлечение распознанных фраз из структуры результата, возвращая их для дальнейшей обработки в системе.

Условие if (!text.empty()) анализирует, содержит ли объект text какую-либо распознанную речевую фразу. Если проверка проходит успешно, это означает, что пользователь произнес осмысленную команду, и система переходит к её обработке, вызывая функцию processCommand(text). Данная логика работы наглядно представлена на рисунке \ref{fig:processCommand_VoiceAssistant}.

\begin{figure}[H]
	\centering
	\includegraphics[scale=0.8]{processCommand_VoiceAssistant.jpg}
	\caption{Метод processCommand()  класса VoiceAssistant}
	\label{fig:processCommand_VoiceAssistant}
\end{figure}

Метод processCommand() выполняет передачу распознанной голосовой команды в менеджер команд для дальнейшей обработки. Он принимает текстовую строку с распознанной речью и у объекта созданного через конструктор который представлен на рисунке \ref{fig:constructor_CommandManager}.


\begin{figure}[H]
	\centering
	\includegraphics[scale=0.8]{constructor_CommandManager.jpg}
	\caption{Конструктор класса CommandManager}
	\label{fig:constructor_CommandManager}
\end{figure}

Конструктор CommandManager при создании объекта регистрирует три голосовые команды для управления оборудованием через метод registerCommand который представлен на рисунке \ref{fig:registerCommand_CommandManager}.

\begin{figure}[H]
	\centering
	\includegraphics[scale=0.8]{registerCommand_CommandManager.jpg}
	\caption{Метод registerCommand()  класса CommandManager}
	\label{fig:registerCommand_CommandManager}
\end{figure}

Метод registerCommand добавляет новую команду в систему управления. Он принимает три параметра: техническое имя команды, голосовую фразу-триггер и обработчик команды. При вызове метод сохраняет команду во внутреннем хранилище commands, связывая между собой имя команды, её голосовой триггер и функцию-обработчик. Техническое имя и голосовой триггер перемещаются в хранилище, а обработчик передается с сохранением категории значения (lvalue или rvalue). Это позволяет регистрировать различные команды с их обработчиками, которые система сможет выполнять при распознавании соответствующих голосовых фраз.

 и после этого созданы объект commandManager вызывает метод process() которые представлен на рисунке \ref{fig:process_CommandManager}.

\begin{figure}[H]
	\centering
	\includegraphics[scale=0.8]{process_CommandManager.jpg}
	\caption{Метод process()  класса CommandManager}
	\label{fig:process_CommandManager}
\end{figure}

Метод CommandManager::process обрабатывает входящую текстовую команду следующим образом. Сначала проверяет, содержит ли текст ключевое слово "алиса" -- если его нет, метод сразу завершает работу без дальнейших действий. Если ключевое слово присутствует, метод перебирает все зарегистрированные команды из хранилища commands и ищет соответствие между текстом команды и сохраненными голосовыми триггерами. При нахождении совпадения метод фиксирует время вызова команды с помощью logTime(), выполняет связанный с командой обработчик handler() и завершает работу. Если ни одна команда не подходит, метод просто завершается без выполнения каких-либо действий. Таким образом, система реагирует только на команды, содержащие слово "алиса" и соответствующие одному из зарегистрированных голосовых шаблонов. Полный код листинга программного обеспечения указан в приложении А.

\newpage
