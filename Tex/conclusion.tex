\section*{Заключение}
\addcontentsline{toc}{section}{Заключение}

В процессе выполнения дипломного проекта на тему «Голосовой помощник для одноплатного компьютера» достигнуты следующие результаты:

\begin{itemize}
	\item Изучены современные методы автоматического распознавания речи, включая традиционные и end-to-end подходы, а также проанализированы их преимущества и ограничения при использовании на устройствах с ограниченными ресурсами;
	\item Исследованы технологии обработки естественного языка, методы взаимодействия с пользовательским голосовым интерфейсом и способы обеспечения автономной работы голосового помощника без подключения к сети Интернет;
	\item Сформулированы требования к функциональности, производительности и ресурсопотреблению программного обеспечения голосового помощника для платформы Raspberry Pi 4 Model B;
	\item Разработана архитектура программного средства, основанная на объектно-ориентированном подходе, с выделением ключевых модулей: распознавания речи, управления командами и обработки аудиоввода;
	\item Реализована программная система на языке программирования C++ с использованием библиотек Vosk и PortAudio, обеспечивающая распознавание голосовых команд и выполнение соответствующих действий в автономном режиме;
	\item Показано, что система способна корректно работать при фоновом шуме, потребляя не более 20\% ресурсов процессора и до 300 МБ оперативной памяти, что соответствует заданным требованиям энергоэффективности и производительности;
	\item Проведено тестирование функциональности и оценки эффективности работы системы, подтверждающие её устойчивость, точность и пригодность для применения в системах «умного дома» и образовательных целях;
	\item Подготовлена техническая документация и программный код, соответствующий требованиям дипломного проектирования.
\end{itemize}

Таким образом, поставленная цель дипломного проектирования достигнута, а предложенное решение может быть использовано в реальных приложениях с ограниченными вычислительными возможностями.


\newpage

\urlstyle{same}
\renewcommand{\refname}{\textbf{Список использованных источников}}
\DeclareFieldFormat{url}{[Электронный ресурс]. -- Режим доступа : \url{#1}}
\DeclareFieldFormat{title}{{#1}}
\DeclareFieldFormat{labelnumberwidth}{#1} % Убирает квадратные скобки

\defbibenvironment{bibliography}
{\list
	{\printfield[labelnumberwidth]{labelnumber}}
	{\setlength{\labelwidth}{1.25cm}%
		\setlength{\leftmargin}{0pt}%
		\setlength{\itemindent}{1.25cm}%
		\setlength{\itemsep}{\bibitemsep}%
		\setlength{\parsep}{\bibparsep}%
		\setlength{\labelsep}{0pt}%
		\setlength{\listparindent}{0pt}}%
	\renewcommand*{\makelabel}[1]{\hspace*{1.25cm}##1\hspace{0.5em}} 
}
{\endlist}
{\item}

\addcontentsline{toc}{section}{Список использованных источников}

\printbibliography


\newpage
