
\section*{Введение}
\addcontentsline{toc}{section}{Введение}




Введение (предисловие) должно быть кратким и четким, не должно быть общих мест и отступлений, непосредственно не связанных с разрабатываемой темой. Объем введения не должен превышать двух страниц.

Рекомендуется следующее содержание введения (предисловия):
\begin{itemize}
	\item краткий анализ достижений в той области, которой посвящена тема дипломного проекта (работы);
	\item цель дипломного проектирования;
	\item принципы, положенные в основу проектирования, научного исследования, поиска технического решения;
	\item обязательное указание задач, решению которых посвящен дипломный проект/работа.
\end{itemize}

Во введении обязательно должна быть отметка о прохождении проверки на антиплагиат с указанием процента оригинальности работы. Рекомендуется добавить в конец введения следующую фразу:

Пояснительную записку выполняют с помощью текстового редактора, используется гарнитура шрифта Times New Roman размером шрифта 14 пунктов с межстрочным интервалом точно 18.

Текст располагают на одной стороне листа формата А4 с соблюдением размеров полей и интервалов, указанных в приложении Л. Абзацы в тексте начинают отступом 1,25, устанавливаемым в Word в диалоговом окне Абзац, (см. приложение Л).

Не допускается использовать в пояснительной записке автоперенос слов. Также рекомендуется заменить букву ё на букву е.

Пояснительная записка должна быть сшита в жестком переплете (специальной папке для дипломных проектов (работ)).

Заключительный абзац введения должен содержать следующее
предложение: «Данный дипломный проект выполнен мной лично, проверен на
заимствования, процент оригинальности составляет ХХ\% (отчет о проверке на
заимствования прилагается).».

\newpage
