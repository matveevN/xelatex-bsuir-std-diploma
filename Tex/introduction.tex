
\section*{Введение}
\addcontentsline{toc}{section}{Введение}

Развитие голосовых технологий в последние годы привело к широкому распространению голосовых помощников в различных сферах -- от потребительской электроники до промышленной автоматизации. Одной из ключевых задач, решаемых при этом, является обеспечение автономности и эффективности работы таких систем на устройствах с ограниченными ресурсами, включая одноплатные компьютеры. Особое внимание в этой области уделяется возможности локального распознавания речи без подключения к сети интернет, что позволяет повысить конфиденциальность и надежность решений.

Целью дипломного проектирования является разработка голосового помощника, работающего в автономном режиме на базе одноплатного компьютера Raspberry Pi 4 Model B. Проект направлен на создание программного обеспечения, обеспечивающего устойчивую и эффективную работу голосового интерфейса в условиях ограниченных вычислительных ресурсов.

В основу проектирования положены принципы модульности, энергоэффективности, а также обеспечения точности и устойчивости распознавания речи при наличии фоновых шумов. В проекте применяются современные технологии обработки речи, включая использование библиотеки Vosk и аудиоплатформы PortAudio, с реализацией на языке программирования C++.

В пояснительной записке рассмотрены основные этапы разработки программного средства. В первом разделе представлен обзор существующих решений и сформулированы требования к системе. Во втором разделе описаны используемые технологии и аппаратная платформа. Третий раздел посвящен архитектуре и реализации голосового помощника, включая описание ключевых модулей. Четвертый раздел содержит результаты тестирования и анализа эффективности системы. Пятый раздел включает технико-экономическое обоснование разработки, в котором произведена оценка затрат и расчет экономической целесообразности.

Дипломная работа выполнена самостоятельно и проверена в системе «Антиплагиат». Процент оригинальности соответствует норме, установленной кафедрой ИКТ, и указан в приложении Б. Цитирования обозначены ссылками на публикации, указанные в списке использованных источников.

\newpage
