\section{Используемые технологии и аппаратная платформа}


\subsection{Язык программирования С++}

Предшественником C++ является язык C, разработанный Деннисом Ритчи в Bell Labs и впервые выпущенный в 1973 году. C -- это широко распространённый язык, который использовался для написания ранних версий Unix и Windows. Действительно, библиотеки и инструменты разработки ПО многих операционных систем до сих пор предоставляют интерфейсы на C.

Сила языка C заключается в том, что он позволяет писать код, который компилируется в компактную форму, использует статическую систему типов, а его типы и структуры данных обеспечивают прямой доступ к памяти компьютера на уровне архитектуры. Именно эти мощные характеристики C стали фундаментом для создания C++~\cite{Cpp}.

С++ является языком программирования общего назначения. Естественная для него область
применения -- системное программирование, понимаемое в широком смысле этого слова. Кроме того, С++ успешно используется во многих областях. Реализации С++ теперь есть на всех машинах, начиная с самых скромных микрокомпьютеров -- до самых больших супер-ЭВМ, и практически для всех операционных систем.

С++ -- язык общего назначения и задуман для того, чтобы настоящие программисты получили
удовольствие от самого процесса программирования. За исключением второстепенных деталей он содержит язык С как подмножество. Язык С расширяется введением гибких и эффективных средств, предназначенных для построения новых типов. Программист структурирует свою задачу, определив новые типы, которые точно соответствуют понятиям предметной области задачи. Такой метод построения программы обычно называют абстракцией данных. Информация о типах содержится в некоторых объектах типов, определенных пользователем. С такими объектами можно работать надежно и просто даже в тех случаях, когда их тип нельзя установить на стадии трансляции. Программирование с использованием таких объектов обычно называют объектно-ориентированным. Если этот метод применяется правильно, то программы становятся короче и понятнее, а сопровождение их упрощается.

Ключевым понятием С++ является класс. Класс -- это определяемый пользователем тип. Классы обеспечивают инкапсуляцию данных, их инициализацию, неявное преобразование пользовательских типов, динамическое задание типов, контролируемое пользователем управление памятью и средства для перегрузки операций. В языке С++ концепции контроля типов и модульного построения программ реализованы более полно, чем в С. Кроме того, С++ содержит усовершенствования, прямо с классами не связанные: символические константы, функции-подстановки, стандартные значения параметров
функций, перегрузка имен функций, операции управления свободной памятью и ссылочный тип. В С++ сохранены все возможности С эффективной работы с основными объектами, отражающими аппаратную "реальность". Это позволяет достаточно эффективно реализовывать пользовательские типы.

Как язык, так и стандартные библиотеки С++ проектировались в расчете на переносимость. Имеющиеся реализации языка будут работать в большинстве систем, поддерживающих С. В программах на С++ можно использовать библиотеки С. Большинство служебных программ, рассчитанных на С, можно использовать и в С++.

Развитие языка С++ происходило на базе языка С, и за небольшим исключением, С был сохранен в качестве подмножества C++. Базовый язык С был спроектирован таким образом, что имеется очень тесная связь между типами, операциями, операторами и объектами, с которыми непосредственно работает машина, т.е. числами, символами и адресами. За исключением операций new, delete и throw, а также проверяемого блока, для выполнения операторов и выражений С++ не требуется скрытой динамической аппаратной или программной поддержки.

В С++ используется та же последовательность команд для вызова функций и возврата из них, что и в С. Если даже эти довольно эффективные операции становятся слишком дорогими, то вызов функции может быть заменен подстановкой ее тела, причем сохраняется удобная функциональная запись безо всяких расходов на вызов функции.

Первоначально язык С задумывался как конкурент ассемблера, способный вытеснить его из основных и наиболее требовательных к ресурсам задач системного программирования. В проекте С++ были приняты меры, чтобы успехи С в этой области не оказались под угрозой. Различие между двумя языками прежде все состоит в степени внимания, уделяемого типам и структурам. Язык С выразителен и в то же время снисходителен по отношению к типам. Язык С++ еще более выразителен, но такой выразительности можно достичь лишь тогда, когда типам уделяют большое внимание. Когда типы объектов известны, транслятор правильно распознает такие выражения, в которых иначе программисту пришлось бы записывать операции с утомительными подробностями. Кроме того, знание типов позволяет транслятору обнаруживать такие ошибки, которые в противном случае были бы выявлены только при тестировании. Отметим, что само по себе использование строгой типизации языка для контроля параметров функции, защиты данных от незаконного доступа, определения новых типов и операций не влечет дополнительных расходов памяти и увеличения времени выполнения программы.

В проекте С++ особое внимание уделяется структурированию программы. Это вызвано увеличением размеров программ со времени появления С. Небольшую программу не более 1000 строк можно заставить работать, нарушая все правила хорошего стиля программирования. Однако, действуя так, человек уже не сможет справиться с большой программой. Если у вашей программы в 10000 строк плохая структура, то вы обнаружите, что новые ошибки появляются в ней так же быстро, как удаляются старые. С++ создавался с целью, чтобы большую программу можно было структурировать таким образом, чтобы одному человеку не пришлось работать с текстом в 25000 строк. В настоящее время можно считать, что эта цель полностью достигнута.

Существуют, конечно, программы еще большего размера. Однако те из них, которые действительно используются, обычно можно разбить на несколько практически независимых частей, каждая из которых имеет значительно меньший упомянутого размер. Естественно, трудность написания и сопровождения программы определяется не только числом строк текста, но и сложностью предметной области. Так что приведенные здесь числа, которыми обосновывались наши соображения, не надо воспринимать слишком серьезно.

К сожалению, не всякую часть программы можно хорошо структурировать, сделать независимой от аппаратуры, достаточно понятной и т.д. В С++ есть средства, непосредственно и эффективно представляющие аппаратные возможности. Их использование позволяет избавиться от беспокойства о надежности и простоте понимания программы. Такие части программы можно скрывать, предоставляя надежный и простой интерфейс с ними.

Естественно, если С++ используется для большой программы, то это означает, что язык используют группы программистов. Полезную роль здесь сыграют свойственные языку модульность, гибкость и строго типизированные интерфейсы. В С++ есть такой же хороший набор средств для создания больших программ, как во многих языках. Но когда программа становится еще больше, проблемы по ее созданию и сопровождению перемещаются из области языка в более глобальную область программных средств и управления проектом~\cite{StroustrupCpp}.


\subsection{Система компиляции}

Языки программирования представляют собой средство описания вычислений для людей и машин. Современный мир зависит от языков программирования, поскольку всё программное обеспечение на всех компьютерах мира написано на том или ином языке программирования. Однако, прежде чем запустить программу, её необходимо преобразовать в форму, которая может выполняться на компьютере.
Программные системы, выполняющие такое преобразование, называются компиляторами~\cite{Dragon}.

Исторически компиляторы не функционируют изолированно, а интегрированы в сложные инструментальные цепочки система компиляции. На практике пользователи редко запускают компилятор напрямую: такие программы, как GCC, Clang или ICC, представляют собой драйверы, координирующие выполнение множества вспомогательных инструментов, включая препроцессор, ассемблер, компоновщик и другие компоненты. Хотя конечной целью является преобразование исходного кода в исполняемый файл, этот процесс требует согласованной работы всего инструментального набора.

Термин "компилятор" в обиходе часто относится не только к самому компилятору, но и к системе компиляции -- обширной экосистеме взаимосвязанных компонентов и библиотек. Для понимания принципов работы компилятора необходимо учитывать его место в этой структуре. Даже в изолированном рассмотрении компилятор выполняет не только оптимизацию кода, но и ряд других задач, что подчеркивает важность понимания архитектуры его внутренних компонентов.

Инструментальная среда разработки представляет собой комплекс программных средств, обеспечивающих полный цикл трансформации исходного кода в исполняемый файл. Ассемблеры преобразуют промежуточный код в машинные инструкции, компоновщики связывают различные объектные модули и библиотеки, а отладчики предоставляют возможности для анализа и исправления ошибок на уровне исходного или машинного кода. Профилировщики, в свою очередь, помогают выявлять узкие места в производительности приложения, позволяя оптимизировать работу программы. В совокупности эти инструменты создают интегрированную цепочку, обеспечивающую не только успешную компиляцию, но и эффективное тестирование, отладку и оптимизацию программного обеспечения. Понимание взаимодействия между этими компонентами критично для создания надежных и высокопроизводительных систем.
Упрощённая схема процесса компиляции представлена на рисунке \ref{fig:tulchain}.

\begin{figure}[H]
	\centering
	\fbox{\includegraphics[scale=0.77]{tul.jpg}}
	\caption{Упрощенная схема системы компиляции}
	\label{fig:tulchain}
\end{figure}

 Ключевыми компонентами системы являются компилятор, ассемблер и линкер. Результатом работы системы является исполняемый файл. Полученный файл может быть отлажен и запущен как на физическом устройстве, так и в среде симуляции. Использование симулятора особенно актуально при кросс-компиляции, когда целевая архитектура отличается от архитектуры системы, на которой выполняется компиляция.

Таким образом, программа-драйвер объединяет три ключевых компонента: компилятор, ассемблер и линкер. Далее рассмотрим каждый из этих элементов более подробно.

Компилятор является центральным и наиболее сложным компонентом системы компиляции. Как видно из рисунка \ref{fig:tulchain}, его основная функция заключается в преобразовании исходного кода, написанного на высокоуровневом языке программирования, в эквивалентный код на языке ассемблера. 

Большинство современных компиляторов функционируют в соответствии с принципом, известным как правило as-if rule. Согласно этому принципу, компилятор может применять любые оптимизации при условии, что наблюдаемое поведение корректно исполняемой программы остаётся эквивалентным поведению, которое предполагал программист при написании исходного кода.

Ключевым аспектом данного правила является то, что оно гарантирует сохранение семантики только для корректных программ. В случае возникновения ситуаций с неопределённым поведением, таких как знаковое целочисленное переполнение, компилятор вправе считать такие сценарии невозможными и не учитывать их при оптимизации.

С архитектурной точки зрения компилятор состоит из frontend и backend. Frontend занимается анализом исходного кода и построением промежуточного представления программы, тогда как backend выполняет оптимизацию этого представления и генерацию машинного кода. Использование англоязычных терминов "фронтенд" и "бэкенд" является общепринятой практикой, поскольку эти понятия прочно укоренились не только в компиляторостроении, но и в других областях компьютерных наук, а попытки их буквального перевода приводят к громоздким и менее точным формулировкам. В некоторых реализациях выделяют дополнительный компонент -- middle-end, ответственный исключительно за оптимизацию, оставляя за backend только финальную кодогенерацию. Однако такая терминология не получила широкого распространения в индустрии, поэтому в дальнейшем изложении мы будем рассматривать backend как единый модуль, включающий как оптимизатор, так и генератор кода.

Термин "ассемблер" имеет два основных значения в контексте систем программирования. Во-первых, это язык ассемблера - низкоуровневый язык программирования, специфичный для конкретной компьютерной архитектуры. Во-вторых, это программа-ассемблер, выполняющая трансляцию исходного кода на языке ассемблера в объектный код.

Объектный код представляет собой промежуточное представление машинных инструкций, где команды процессора уже закодированы в двоичной форме, но ещё не готовы к непосредственному исполнению. Такой код требует дополнительной обработки линкером для формирования окончательного исполняемого файла.

В современных системах компиляции ассемблер может быть реализован как отдельная программа или интегрирован в компилятор. В последнем случае компилятор способен генерировать объектный код напрямую, минуя этап явного ассемблирования. Такая интеграция позволяет оптимизировать процесс трансляции и повысить общую производительность системы компиляции.

Возникает закономерный вопрос: зачем нужен язык ассемблера, если компилятор может генерировать объектный код напрямую. Это обусловлено несколькими важными причинами.

Во-первых, ассемблерный код значительно удобнее для анализа и отладки по сравнению с бинарным представлением. При диагностике сложных проблем, особенно тех, которые слабо проявляются в отладочных сборках, работа с дизассемблированным кодом остаётся стандартной практикой.

Во-вторых, набор машинных инструкций современных процессоров гораздо шире, чем те, которые обычно генерирует компилятор. Многие специализированные команды  для работы с системными регистрами или управления аппаратными особенностями процессора недоступны через высокоуровневые языки. В таких случаях при разработке низкоуровневого ПО, включая операционные системы или драйверы написание кода на ассемблере становится необходимым.

Таким образом, несмотря на возможность прямой генерации объектного кода, язык ассемблера сохраняет свою актуальность как для отладки, так и для низкоуровневого программирования.

Ещё один закономерный вопрос -- зачем нужен объектный код, если можно сразу генерировать исполняемый файл. Основная причина кроется в необходимости раздельной компиляции. Практически все современные языки программирования поддерживают возможность компиляции отдельных модулей независимо друг от друга. Это особенно важно при работе с библиотеками -- их код компилируется один раз, а затем многократно используется в разных проектах. Объектный файл как раз и служит таким промежуточным форматом, который сохраняет не только сам машинный код, но и всю необходимую для линкера информацию: таблицы символов, неразрешённые внешние ссылки, данные для отладки и другую метаинформацию. При этом в итоговом исполняемом файле большая часть этих данных становится ненужной -- линкер уже разрешил все зависимости и создал единый исполняемый образ. Таким образом, объектный код выступает в роли универсального контейнера, который с одной стороны содержит всё необходимое для последующей компоновки, а с другой -- позволяет избежать повторной компиляции одних и тех же модулей, значительно ускоряя процесс разработки.

Статическая линковка выполняется после ассемблирования. В простейшем случае достаточно передать линкеру объектные файлы вместе со стандартными библиотеками. Однако пути к этим библиотекам сильно зависят от параметров сборки: используются разные версии для 32-битного и 64-битного кода, различные ABI в рамках одной архитектуры и другие специфические варианты целевой платформы. Поэтому правильные пути к библиотекам обычно определяет драйвер компилятора на основе множества факторов, включая неочевидные системные настройки. Хотя эти пути можно вручную подсмотреть в выводе компилятора, явно задавать их вручную крайне не рекомендуется, так как это может привести к тонким ошибкам совместимости. Вместо этого следует полагаться на автоматическое определение путей, предоставляемое инструментарием сборки \cite{Vladimirov2024}.

\subsection{Генератор систем сборки CMake}

Если вам когда-либо приходилось поддерживать процесс сборки и установки программного пакета, вас наверняка заинтересует CMake. CMake -- это генератор систем сборки с открытым исходным кодом, который позволяет разработчикам задавать параметры сборки в простом, переносимом текстовом формате. Этот файл затем используется CMake для генерации проектных файлов под нативные инструменты сборки, включая интегрированные среды разработки, такие как Microsoft Visual Studio или Apple Xcode, а также UNIX, Linux, NMake и Ninja. CMake упрощает сложные аспекты сборки программного обеспечения, такие как кроссплатформенная сборка, анализ системы и настройка под пользовательские требования, предоставляя удобные средства для адаптации сборки под сложные аппаратные и программные системы \cite{CMake}.

Без системы сборки проект -- это просто набор файлов. CMake вносит в этот хаос порядок, начиная с человекочитаемого файла под названием CMakeLists.txt. Этот файл определяет, что можно собрать и как, какие тесты можно запустить, а также пакеты для создания. Это платформонезависимое описание всего проекта, которое CMake затем преобразует в специфичные для платформы файлы проектов для инструментов сборки. Как следует из названия, это обычный текстовый файл, который разработчики редактируют в своём любимом текстовом редакторе или среде разработки. Содержимое этого файла подробно рассматривается в последующих главах, но пока достаточно знать, что именно он управляет всеми действиями CMake при настройке и выполнении сборки.

Одна из фундаментальных концепций CMake -- разделение проекта на исходный и бинарный каталоги. Исходный каталог -- это место, где находится файл CMakeLists.txt, а также исходные файлы проекта и все остальные файлы, необходимые для сборки, организованные внутри этой директории. Исходный каталог часто находится под управлением системы контроля версий, такой как Git, Subversion или аналогичной \cite{Professional_CMake}.

С 1999 года CMake активно развивается и достиг уровня, когда он стал проверенным решением для широкого круга задач сборки. Разработка CMake началась в рамках проекта Insight Toolkit, это крупный программный проект, который должен работать на множестве платформ и взаимодействовать с другими программными пакетами. Чтобы обеспечить это, потребовался мощный, но простой в использовании инструмент сборки. Опираясь на опыт работы с системами сборки для больших проектов, разработчики создали CMake, учитывая эти потребности. С тех пор популярность CMake неуклонно растёт, и многие проекты и разработчики выбирают его за простоту и гибкость. Наиболее яркий пример -- успешное внедрение CMake в качестве системы сборки K Desktop Environment, одного из крупнейших проектов с открытым исходным кодом.

Для любого проекта, особенно кроссплатформенного, необходима единая система сборки. Многие проекты, не использующие CMake, поставляются как с Makefile для UNIX, так и с рабочей областью Microsoft Visual Studio. Это вынуждает разработчиков постоянно поддерживать обе системы сборки в актуальном и согласованном состоянии. Добавление поддержки других систем, например Xcode, требует ещё большего количества кастомных файлов, усугубляя проблему. Ситуация становится ещё сложнее, если нужно поддерживать опциональные компоненты, такие как подключение JPEG-поддержки при наличии libjpeg в системе. CMake решает эту проблему, объединяя все эти операции в одном простом и понятном формате файла.

CMake также включает поддержку тестирования программного обеспечения через CTest. Часть процесса тестирования включает сборку, возможную установку и определение того, какие компоненты программного обеспечения подходят для текущей системы. Это делает CTest логичным расширением CMake, поскольку он уже содержит большую часть этой информации. Аналогичным образом, CMake включает CPack -- инструмент для кросс-платформенного распространения программного обеспечения. Он предоставляет универсальный способ создания нативных установщиков, используя популярные форматы, такие как WiX, RPM, Cygwin и PackageMaker.

Если над проектом работает несколько разработчиков или он предназначен для нескольких целевых платформ, сборка будет выполняться на разных компьютерах. Учитывая разнообразие установленного ПО и пользовательских настроек современных систем, даже две машины с одной ОС могут иметь различия. CMake предоставляет множество преимуществ для одноплатформенных, но многомашинных сред разработки, включая:
\begin{itemize}
	\item автоматический поиск зависимостей;
	\item отдельная директория сборки;
	\item генерация исходного кода;
	\item выбор компонентов;
	\item генерация проектов;
	\item параллельная сборка.
\end{itemize}

CMake продолжает поддерживать новые инструменты сборки по мере их появления. Он быстро добавляет совместимость с новыми версиями Microsoft Visual Studio и Apple Xcode. Кроме того, в CMake была добавлена поддержка Ninja -- современного инструмента сборки от Google. Благодаря CMake, после написания конфигурационных файлов вы автоматически получаете поддержку новых компиляторов и систем сборки, так как она встроена в новые версии CMake и не зависит от вашего дистрибутива. CMake также обеспечивает кросс-компиляцию для других операционных систем или встраиваемых устройств. Большинство команд в CMake корректно обрабатывают различия между хост-системой и целевой платформой при кросс-компиляции \cite{CMake}.

\subsection{Система контроля версий Git}

Говоря простыми словами, Git -- это система отслеживания содержимого, в ней используются те же принципы, что и в большинстве систем контроля версий. Тем не менее уникальной Git делает ее распределенная природа. Благодаря этому система быстрая и масштабируемая, имеет множество наборов команд, дающих доступ к высокоуровневым и низкоуровневым операциям, а также оптимизирована для выполнения локальных операций \cite{Git}.

Многие люди в качестве метода контроля версий применяют копирование файлов в
отдельный каталог. Данный подход очень распространён из-за его простоты, однако он невероятно сильно подвержен появлению ошибок. Можно легко забыть в каком каталоге вы находитесь и случайно изменить не тот файл или скопировать не те файлы, которые вы хотели.

Как и многие вещи в жизни, Git начинался с капелькой творческого хаоса и бурных споров.

Ядро Linux -- это достаточно большой проект с открытым исходным кодом. Большую часть времени разработки ядра Linux (1991–2002 гг.) изменения передавались между разработчиками в виде патчей и архивов. В 2002 году проект ядра Linux начал использовать проприетарную децентрализованную систему контроля версий BitKeeper.

В 2005 году отношения между сообществом разработчиков ядра Linux и коммерческой компанией, которая разрабатывала BitKeeper, прекратились, и бесплатное использование утилиты стало невозможным. Это сподвигло сообщество разработчиков ядра Linux (в частности Линуса Торвальдса -- создателя Linux) разработать свою собственную утилиту, учитывая уроки, полученные при работе с BitKeeper. Некоторыми целями, которые преследовала новая система, были:
\begin{itemize}
	\item скорость;
	\item простая архитектура;
	\item эффективную работу с разветвлёнными проектами;
	\item полная децентрализация;
	\item возможность эффективного управления большими проектами.
\end{itemize}

С момента своего появления в 2005 году, Git развился в простую в использовании систему, сохранив при этом свои изначальные качества. Он удивительно быстр, эффективен в работе с большими проектами и имеет великолепную систему веток для нелинейной разработки.

Основное отличие Git от любой другой системы контроля версий это подход к работе со своими данными. Концептуально, большинство других
систем хранят информацию в виде списка изменений в файлах. Эти системы (CVS,
Subversion, Perforce, Bazaar и т. д.) представляют хранимую информацию в виде набора файлов и изменений, сделанных в каждом файле, по времени.

Git не хранит и не обрабатывает данные таким способом. Вместо этого, подход Git к хранению данных больше похож на набор снимков миниатюрной файловой системы. Каждый раз, когда вы делаете коммит, то есть сохраняете состояние своего проекта в Git, система запоминает, как выглядит каждый файл в этот момент, и сохраняет ссылку на этот снимок. Для увеличения эффективности, если файлы не были изменены, Git не запоминает эти файлы вновь, а только создаёт ссылку на предыдущую версию идентичного файла, который уже сохранён. Git представляет свои данные как, скажем, поток снимков.

Это очень важное различие между Git и почти любой другой системой контроля версий. Git переосмысливает практически все аспекты контроля версий, которые были скопированы из предыдущего поколения большинством других систем. Это делает Git больше похожим на миниатюрную файловую систему с удивительно мощными утилитами, надстроенными над ней, нежели просто на VCS. Когда мы будем рассматривать управление ветками в главе Ветвление в Git, мы увидим, какие преимущества вносит такой подход к работе с данными в Git.

Для работы большинства операций в Git достаточно локальных файлов и ресурсов --  в основном, системе не нужна никакая информация с других компьютеров в вашей сети. Если вы привыкли к централизованным системам контроля версий, где большинство операций страдают от задержек из-за работы с сетью, то этот аспект Git заставит вас думать, что боги скорости наделили Git несказанной мощью. Так как вся история проекта хранится прямо на вашем локальном диске, большинство операций кажутся чуть ли не мгновенными.

Для примера, чтобы посмотреть историю проекта, Git не нужно соединяться с сервером для её получения и отображения -- система просто считывает данные напрямую из локальной базы данных. Это означает, что вы увидите историю проекта практически моментально. Если вам необходимо посмотреть изменения, сделанные между текущей версией файла и версией, созданной месяц назад, Git может найти файл месячной давности и локально вычислить изменения, вместо того, чтобы запрашивать удалённый сервер выполнить эту операцию, либо вместо получения старой версии файла с сервера и выполнения операции локально.

Это также означает, что есть лишь небольшое количество действий, которые вы не сможете выполнить, если вы находитесь оффлайн или не имеете доступа к VPN в данный момент. Если вы в самолёте или в поезде и хотите немного поработать, вы сможете создавать коммиты без каких-либо проблем и когда будет возможность подключиться к сети, все изменения можно будет синхронизировать. Если вы ушли домой и не можете подключиться через VPN, вы всё равно сможете работать.
Добиться такого же поведения во многих других системах либо очень сложно, либо вовсе невозможно. В Perforce, для примера, если вы не подключены к серверу, вам не удастся сделать многого; в Subversion и CVS вы можете редактировать файлы, но вы не сможете сохранить изменения в базу данных (потому что вы не подключены к БД). Всё это может показаться не таким уж и значимым, но вы удивитесь, какое большое значение это может
иметь.

В Git для всего вычисляется хеш-сумма, и только потом происходит сохранение. В
дальнейшем обращение к сохранённым объектам происходит по этой хеш-сумме. Это
значит, что невозможно изменить содержимое файла или каталога так, чтобы Git не узнал об этом. Данная функциональность встроена в Git на низком уровне и является неотъемлемой частью его философии. Вы не потеряете информацию во время её передачи и не получите повреждённый файл без ведома Git \cite{ProGit}.

\subsection{Одноплатный компьютер Raspberry Pi 4 Model B}

Raspberry Pi представляет собой одноплатный компьютер, что буквально отражает его суть: это полнофункциональная вычислительная система, аналогичная настольным ПК, ноутбукам или смартфонам, но реализованная на единой печатной плате. Как и другие представители класса одноплатных компьютеров, Raspberry Pi отличается исключительной компактностью -- его размеры сопоставимы с банковской картой. Однако это не означает ограниченную производительность: устройство сохраняет полную функциональность традиционных компьютеров, отличаясь лишь несколько меньшей скоростью выполнения задач.

Семейство одноплатных компьютеров Raspberry Pi было создано в рамках образовательной инициативы некоммерческой организации Raspberry Pi Foundation, ставившей целью сделать изучение компьютерных технологий более доступным и практико-ориентированным. Изначально основатели проекта не ожидали такого масштаба популярности -- первая пробная партия из нескольких тысяч устройств, выпущенная в 2012 году, была распродана моментально. За прошедшие годы эти компактные компьютеры получили поистине глобальное распространение -- совокупные продажи исчисляются десятками миллионов экземпляров. Область применения Raspberry Pi сегодня чрезвычайно широка: от образовательных учреждений и домашнего использования до промышленных предприятий, центров обработки данных и даже экстремальных условий. Их можно встретить в составе автономных морских судов и высотных стратосферных аппаратов.

С момента выпуска первой модели Raspberry Pi Model B было разработано значительное количество модификаций данной платформы, каждая из которых либо обладала улучшенными техническими характеристиками, либо была оптимизирована для решения специфических задач. В качестве примера можно привести линейку Raspberry Pi Zero, представляющую собой уменьшенную версию базовой платы. Данная модификация пожертвовала частью функциональности (такой как наличие нескольких USB-портов и Ethernet-разъема) в пользу уменьшения габаритов и снижения энергопотребления.

Важной особенностью всех моделей Raspberry Pi является их полная программная совместимость. Разработанное программное обеспечение для одной модели может быть запущено на любой другой версии платы. Более того, даже самая современная версия операционной системы Raspberry Pi OS сохраняет совместимость с первоначальной моделью Model B, выпущенной до официального старта продаж, хотя и с заметным снижением производительности.

В отличие от традиционных компьютерных систем, компоненты которых скрыты внутри корпуса, архитектура Raspberry Pi предусматривает открытое расположение всех электронных компонентов, интерфейсных разъемов и портов ввода-вывода. Такая конструктивная особенность делает данную платформу исключительно удобной для образовательных целей, позволяя наглядно изучать принципы построения компьютерных систем. Кроме того, открытая компоновка существенно упрощает процесс подключения периферийных устройств, что способствует более быстрому освоению платформы.

На рисунке \ref{fig:raspberry4B} представлен внешний вид сверху одноплатного компьютера Raspberry Pi 4 Model B. Несмотря на кажущуюся сложность компоновки, архитектура платы имеет четкую логическую структуру. Рассмотрим основные функциональные компоненты устройства.

\begin{figure}[H]
	\centering
	\fbox{\includegraphics[scale=1.1]{raspberryPi.jpg}}
	\caption{Raspberry Pi 4 Model B}
	\label{fig:raspberry4B}
\end{figure}

Архитектура Raspberry Pi, как и любого современного компьютера, основана на взаимодействии специализированных компонентов, каждый из которых выполняет строго определённые функции. Ключевым элементом системы является SoC, расположенная в центральной части платы под металлическим теплораспределительным корпусом, что показано на рисунке~\ref{fig:soc}. 

\begin{figure}[H]
	\centering
	\fbox{\includegraphics[scale=0.9]{systemOnChip.jpg}}
	\caption{Raspberry Pi’s system-on-chip (SoC)}
	\label{fig:soc}
\end{figure}

Концепция "системы на кристалле" (SoC) представляет собой высокоинтегрированное решение, где все основные компоненты вычислительной системы объединены в рамках одного полупроводникового кристалла. В отличие от классических компьютерных архитектур с раздельными микросхемами, здесь CPU,  GPU, модули памяти, контроллеры ввода-вывода и различные специализированные ускорители располагаются на единой подложке. Такая компоновка обеспечивает значительное сокращение физических размеров системы, снижение энергопотребления и повышение общей производительности за счет минимизации задержек при передаче данных между компонентами.

Архитектура SoC особенно востребована в мобильных устройствах и встраиваемых системах, где критически важны компактность и энергоэффективность. Современные системы на кристалле представляют собой сложные гетерогенные вычислительные платформы, где помимо традиционных CPU и GPU интегрированы дополнительные специализированные блоки нейропроцессоры для задач ИИ, аппаратные ускорители криптографии и другие узкоспециализированные модули. Такая модульная архитектура позволяет создавать гибкие решения, оптимизированные под конкретные прикладные задачи, будь то компьютерное зрение, обработка мультимедиа или работа с беспроводными интерфейсами.

Важной особенностью современных SoC является использование передовых технологий проектирования, включая трехмерную компоновку компонентов и технологию чиплетов, когда отдельные функциональные блоки изготавливаются по разным техпроцессам и затем интегрируются в единую систему. Это позволяет достичь оптимального баланса между производительностью, энергопотреблением и стоимостью. 

Для работы вычислительной системы необходима память разных типов. Рядом с SoC расположен модуль RAM -- компактный черный чип квадратной формы смотрите рисунок \ref{fig:ram}.

\begin{figure}[H]
	\centering
	\fbox{\includegraphics[scale=0.9]{ram.jpg}}
	\caption{Raspberry Pi’s random access memory (RAM)}
	\label{fig:ram}
\end{figure}

 RAM временно хранит данные и команды, с которыми процессор работает в текущий момент, обеспечивая высокоскоростной доступ к информации. Её ключевая особенность -- энергозависимость: при отключении питания все хранящиеся в RAM данные безвозвратно теряются. Это связано с технологией  DRAM, где информация хранится в виде зарядов в микроскопических конденсаторах, которые постепенно разряжаются и требуют постоянного обновления.

В отличие от RAM, microSD-карта использует энергонезависимую флеш-память (NAND), сохраняющую данные даже при полном отключении питания. Однако за это постоянство приходится платить: скорость записи и чтения у microSD на порядок ниже, чем у оперативной памяти. Кроме того, флеш-память имеет ограниченный ресурс циклов перезаписи (обычно 10 000-100 000 циклов для разных типов ячеек).

Эти два типа памяти образуют иерархическую систему хранения данных в современных устройствах. Процессор сначала загружает необходимые данные с microSD в оперативную память, где может быстро их обрабатывать. Все изменения сначала вносятся в RAM, и только при явной команде записи (или автоматически через определенные промежутки времени) сохраняются на microSD. Такая архитектура позволяет сочетать преимущества высокой скорости оперативной памяти и постоянного хранения флеш-памяти.

В правом верхнем углу платы расположен радиомодуль, скрытый под металлическим экраном смотрите рисунок \ref{fig:radio_module}), который обеспечивает беспроводную связь Raspberry Pi.

\begin{figure}[H]
	\centering
	\fbox{\includegraphics[scale=0.9]{radioModule.jpg}}
	\caption{Raspberry Pi’s radio module}
	\label{fig:radio_module}
\end{figure}

Этот компактный модуль представляет собой комбинированное решение для беспроводной связи, объединяющее Wi-Fi и Bluetooth технологии в едином корпусе. Wi-Fi адаптер обеспечивает высокоскоростное подключение к беспроводным сетям, поддерживая современные стандарты 802.11ac/ax, что позволяет достигать скорости передачи данных до нескольких гигабит в секунду в диапазонах 2.4 ГГц и 5 ГГц. Bluetooth-радио модуля, обычно версии 5.0 или выше, отвечает за подключение периферийных устройств -- беспроводных наушников, клавиатур, мышей, а также обеспечивает связь с умными устройствами экосистемы "умного дома". Металлический экран, окружающий модуль, играет критически важную роль -- он эффективно подавляет электромагнитные помехи, предотвращая их влияние на чувствительные радиокомпоненты и одновременно ограничивая излучение от самого модуля, что особенно важно для соблюдения норм электромагнитной совместимости. Такая конструкция обеспечивает стабильную работу беспроводных интерфейсов даже в условиях плотного скопления электронных устройств, гарантируя надежное соединение без разрывов и помех. Современные версии подобных модулей часто поддерживают дополнительные функции вроде одновременной работы в двух диапазонах (Dual-Band), технологии формирования луча (Beamforming) и энергоэффективные протоколы Bluetooth Low Energy, что делает их универсальным решением для широкого спектра устройств -- от смартфонов до систем "умного дома".


В нижней части платы расположен USB-контроллер (черная микросхема в пластиковом корпусе), который управляет работой четырех USB-портов. Рядом с ним находится компактный сетевой контроллер, отвечающий за функционирование Ethernet-порта. В левом верхнем углу платы, непосредственно над разъемом питания USB Type-C, размещена интегральная схема управления питанием (PMIC) смотрите рисунок \ref{fig:pmic}.

\begin{figure}[H]
	\centering
	\fbox{\includegraphics[scale=0.9]{pmic.jpg}}
	\caption{Raspberry Pi’s power management integrated circuit (PMIC)}
	\label{fig:pmic}
\end{figure}

Этот критически важный компонент выполняет преобразование входного напряжения от источника питания в стабильные уровни, необходимые для корректной работы всех элементов Raspberry Pi. PMIC представляет собой сложную систему управления питанием, которая не просто преобразует напряжения, но и интеллектуально распределяет энергопотребление между различными узлами системы в зависимости от текущей нагрузки.

Микросхема PMIC обеспечивает надежное электропитание процессора, оперативной памяти и периферийных устройств, поддерживая стабильность работы системы в различных режимах эксплуатации -- от минимального энергопотребления в режиме ожидания до пиковых нагрузок при выполнении ресурсоемких задач. Она включает в себя несколько импульсных стабилизаторов напряжения (DC-DC converters) и линейных регуляторов (LDO), каждый из которых оптимизирован для питания определенного компонента системы с требуемыми параметрами по току и уровню пульсаций.

Особенностью PMIC в Raspberry Pi является способность динамически адаптироваться к изменениям входного напряжения и тока, обеспечивая защиту от перегрузок, перегрева и коротких замыканий. Компонент также отвечает за последовательность включения различных подсистем (power sequencing), что критически важно для стабильной работы сложной электроники. Современные версии PMIC дополнительно реализуют функции энергоменеджмента, позволяя оптимизировать энергопотребление системы в целом.

На плате Raspberry Pi расположены четыре порта USB, которые можно увидеть в центральной и правой части нижнего края рисунок \ref{fig:usb}.

\begin{figure}[H]
	\centering
	\fbox{\includegraphics[scale=0.92]{usb.jpg}}
	\caption{Raspberry Pi's USB ports}
	\label{fig:usb}
\end{figure}

Эти универсальные порты обеспечивают подключение широкого спектра USB-устройств -- от базовой периферии вроде клавиатур и мышей до высокоскоростных внешних накопителей, цифровых камер и специализированного оборудования. Как видно на рисунке \ref{fig:usb}, плата оснащена двумя типами портов: традиционными USB 2.0 с характерными черными внутренними разъемами (пропускная способность до 480 Мбит/с) и современными USB 3.0 с синими компонентами (скорость до 5 Гбит/с).

Такая комбинация обеспечивает полную совместимость как с устаревшими устройствами, так и с современными гаджетами, требующими высокой пропускной способности. Особенностью реализации является интеллектуальное распределение питания -- каждый порт может обеспечивать до 1.5А тока, что позволяет подключать энергоемкие устройства без внешнего питания.

Конструкция портов предусматривает защиту от перегрузок и коротких замыканий, что значительно повышает надежность системы. Кроме того, расположение разъемов оптимизировано для удобного доступа, даже если плата установлена в компактном корпусе. Для пользователей важна и обратная совместимость -- устройства USB 3.0 корректно работают и в портах USB 2.0, хотя и с ограничением скорости.

На плате Raspberry Pi справа от USB-портов расположен Ethernet-разъём (рисунок \ref{fig:ethernet}), предназначенный для подключения к проводной сети.

\begin{figure}[H]
	\centering
	\fbox{\includegraphics[scale=0.95]{ethernet.jpg}}
	\caption{Raspberry Pi’s Ethernet port}
	\label{fig:ethernet}
\end{figure}

Этот стандартный сетевой интерфейс использует разъём типа RJ45, совместимый с обычными сетевыми кабелями. В нижней части порта, как показано на рисунке \ref{fig:ethernet}, расположены два светодиодных индикатора (LED), которые визуально отображают состояние сетевого подключения: активность передачи данных и наличие физического соединения. Такой интерфейс обеспечивает стабильное и высокоскоростное сетевое соединение, что особенно важно для серверных и сетевых применений Raspberry Pi.

На левой стороне платы Raspberry Pi, чуть выше USB-портов, расположен комбинированный 3.5 мм аудио-видео разъём (рисунок \ref{fig:jack}).

\begin{figure}[H]
	\centering
	\fbox{\includegraphics[scale=0.95]{jack.jpg}}
	\caption{Raspberry Pi’s 3.5 mm AV jack}
	\label{fig:jack}
\end{figure}

Основное его назначение -- вывод аналогового аудиосигнала, при этом подключение к активным колонкам обеспечивает лучшее качество звука, чем использование наушников. Кроме того, через этот разъём возможна передача аналогового видеосигнала (композитного видео) на телевизоры и проекторы при использовании специального TRRS-кабеля. Такой функционал делает данный интерфейс универсальным решением для мультимедийных применений.

Над комбинированным 3.5 мм аудио-видео разъёмом расположен специализированный интерфейс CSI с пластиковым фиксатором (рисунок~\ref{fig:camera}).

\begin{figure}[H]
	\centering
	\fbox{\includegraphics[scale=0.95]{camera.jpg}}
	\caption{Raspberry Pi’s camera connector}
	\label{fig:camera}
\end{figure}

Этот разъём предназначен для подключения официального модуля камеры Raspberry Pi Camera Module, который подробно рассматривается в главе 8 настоящего руководства. Данный интерфейс использует высокоскоростной последовательный протокол MIPI CSI-2, специально разработанный для цифровых камер и обеспечивающий передачу видеоданных с минимальными задержками. Разъём имеет 15-контактную конструкцию, включающую два канала передачи видеоданных, линии управления и синхронизации, отдельное питание для камеры и I2C-интерфейс для конфигурации. Пластиковая защёлка с фиксатором надёжно удерживает гибкий плоский кабель, предотвращая случайное отсоединение даже при вибрациях. Конструкция разъёма рассчитана на многократные подключения (до 10 000 циклов) и имеет ключевую метку для правильной ориентации кабеля. Совместимость с различными версиями камерных модулей Raspberry Pi обеспечивается благодаря стандартизированному расположению контактов и стабильным электрическим характеристикам интерфейса. Разъём поддерживает разрешение видео до 4K при частоте 30 кадров в секунду, что делает его идеальным решением для проектов компьютерного зрения, видеонаблюдения и машинного обучения.


В верхней части левого края платы расположены micro-HDMI разъемы (рисунок \ref{fig:microHDMI}) -- компактные версии стандартных HDMI-интерфейсов.

\begin{figure}[H]
	\centering
	\fbox{\includegraphics[scale=0.93]{microHDMI.jpg}}
	\caption{Raspberry Pi’s micro-HDMI ports}
	\label{fig:microHDMI}
\end{figure}

Как видно на рисунке \ref{fig:microHDMI}, данные порты обеспечивают передачу цифровых аудио и видеосигналов высокой четкости, что позволяет подключать Raspberry Pi к одному или двум внешним устройствам отображения (мониторам, телевизорам или проекторам). Особенностью micro-HDMI является сохранение всех функциональных возможностей полноразмерного HDMI-интерфейса при существенно меньших габаритах разъема.


В верхней части платы, над HDMI-портами, расположен разъём питания USB Type-C (рисунок \ref{fig:typeC}).

\begin{figure}[H]
	\centering
	\fbox{\includegraphics[scale=0.93]{typeC.jpg}}
	\caption{Raspberry Pi’s USB Type-C power port}
	\label{fig:typeC}
\end{figure}

Данный интерфейс, широко распространённый в современных мобильных устройствах, служит для подключения Raspberry Pi к источнику электропитания. Как показано на рисунке \ref{fig:typeC}, этот компактный разъём обеспечивает надёжное соединение и позволяет использовать различные источники питания. Однако для стабильной работы системы, особенно при высокой нагрузке, рекомендуется применять официальный блок питания Raspberry Pi, специально разработанный для данной платформы и обеспечивающий необходимые параметры тока и напряжения.


В верхней части платы расположен специализированный разъём DSI смотрите рисунок \ref{fig:display}, который внешне напоминает разъём для камеры, но функционально является его противоположностью.

\begin{figure}[H]
	\centering
	\fbox{\includegraphics[scale=0.93]{display.jpg}}
	\caption{Raspberry Pi’s display connector (DSI)}
	\label{fig:display}
\end{figure}

Как показано на рисунке \ref{fig:gpio}, универсальный 40-контактный GPIO обеспечивает гибкое подключение широкого спектра внешних устройств -- от простых электронных компонентов до сложных датчиков и измерительных приборов. Этот программируемый интерфейс включает цифровые входы/выходы с настраиваемыми режимами работы, поддерживает стандартные протоколы передачи данных (I2C, SPI, UART), а также предоставляет аналоговые входы (при использовании внешнего АЦП) и ШИМ-выходы для управления сервоприводами и светодиодами. Функциональные возможности GPIO, включая назначение контактов и примеры практического применения, подробно рассматриваются в главе 6 настоящего руководства. Особенностью данного интерфейса является возможность создания аппаратных прерываний, что позволяет реализовывать системы реального времени с минимальной задержкой отклика. Совместимость с большинством популярных датчиков и модулей расширения делает GPIO ключевым элементом для разработки проектов интернета вещей, робототехники и автоматизации.

Пластиковый фиксатор с защёлкой обеспечивает надёжное соединение с гибким плоским кабелем дисплейного модуля, предотвращая его случайное отсоединение при механических воздействиях. Интерфейс поддерживает разрешение до 1920×1080  при 60 кадрах в секунду, что позволяет реализовать плавное отображение графики и сенсорного ввода. Благодаря встроенной поддержке сенсорного интерфейса через отдельные контакты I2C, решение не требует дополнительных периферийных подключений, что делает его идеальным для компактных встраиваемых систем с интерактивным управлением. Совместимость с официальными дисплеями Raspberry Pi гарантируется точным соответствием электрических характеристик и протокола обмена данными.

На правой границе платы расположен 40-контактный GPIO-разъём, представляющий собой два параллельных ряда металлических контактов (рисунок \ref{fig:gpio}).

\begin{figure}[H]
	\centering
	\fbox{\includegraphics[scale=0.93]{gpio.jpg}}
	\caption{Raspberry Pi’s GPIO header}
	\label{fig:gpio}
\end{figure}

Как показано на рисунке \ref{fig:gpio}, данный интерфейс обеспечивает возможность подключения широкого спектра внешних устройств -- от простых электронных компонентов  до сложных датчиков и измерительных приборов. 

Непосредственно под GPIO-разъёмом, со смещением влево, расположен дополнительный 4-контактный интерфейс (рисунок \ref{fig:gpio}), предназначенный для подключения модуля PoE HAT (Power over Ethernet). Как видно на рисунке \ref{fig:gpio}, данный модуль позволяет организовать электропитание Raspberry Pi через Ethernet-кабель, что обеспечивает альтернативу стандартному питанию через USB Type-C разъём и особенно востребовано в сетевых и встраиваемых решениях.

На обратной стороне платы Raspberry Pi, прямо под дисплейным разъёмом, расположен слот для microSD-карты (рисунок \ref{fig:microSd}).

\begin{figure}[H]
	\centering
	\fbox{\includegraphics[scale=0.93]{microSd.jpg}}
	\caption{Raspberry Pi’s microSD card connector}
	\label{fig:microSd}
\end{figure}

Этот слот служит основным хранилищем системы -- в него вставляется карта памяти с операционной системой, пользовательскими файлами и установленными программами. В отличие от оперативной памяти, данные на microSD-карте сохраняются после выключения питания. Использование съёмного носителя обеспечивает гибкость при смене операционной системы или восстановлении данных \cite{RaspberryPi}.

Рекомендуемый диапазон рабочей температуры окружающей среды составляет от 0 до 50 градусов Цельсия.

Чтобы снизить тепловыделение в режиме простоя или при небольшой нагрузке, Raspberry Pi 4 Model B уменьшает тактовую частоту и напряжение процессора. При более высокой нагрузке частота и напряжение (а следовательно, и тепловыделение) увеличиваются. Встроенный регулятор автоматически снижает тактовую частоту и напряжение процессора, чтобы его температура никогда не превышала 85 градусов Цельсия.

Raspberry Pi 4 Model B отлично работает без дополнительного охлаждения и рассчитан на кратковременную высокую производительность, предполагается что в среднем нагрузка будет небольшой, а частота процессора будет повышаться только при необходимости (например, при загрузке веб-страницы). Если пользователь планирует постоянно нагружать систему или эксплуатировать её при высокой температуре с максимальной производительностью, может потребоваться дополнительное охлаждение~\cite{RaspberryPi_DATASHEET}.




\newpage
