\section{Анализ литературных источников и формирование требований к проектируемому программному обеспечению}

\subsection{Аналитический обзор литературных источников}

В современном мире наблюдаются устойчивые тенденции, связанные с развитием и интеграцией в различных бытовых и профессиональных сферах жизнедеятельности технологий распознавания голоса. На основе данных инструментов представляется возможность взаимодействия между человеком и цифровым устройством исключительно на основе использования голоса. Использование методов распознавания голоса является основой работы современных голосовых помощников, способных упростить и качественно улучшить жизнь современного человека. Наиболее распространёнными из примеров использования данных технологий являются индивидуальные голосовые помощники, телефонные роботы и ряд иных задач, при выполнении которых
происходит экономия временных ресурсов человека~\cite{Hein}.

Помимо этого, в зависимости от используемых методов распознавания голоса, активно разрабатываются интеллектуальные системы обучения, способные принимать экзамен и помогать в изучении иностранных языков. При этом использование голосовых помощников получает свое развитие и в решении наиболее сложных задач
из профессиональных сфер человеческой жизнедеятельности. Примерами данного применения являются – идентификация личности, судебная экспертиза, помощь людям с ограниченными возможностями и ряд иных задач~\cite{Muravyov}.

Широкое распространение голосовых помощников стало возможным благодаря развитию методов распознавания голоса. На сегодняшний день существует два основных направления развития по распознаванию речи – традиционное направление автоматического распознавания речи и направление комплексного (end-to-end)
подхода к распознаванию речи~\cite{Riqiang}. При этом необходимо отметить, что каждый из данных методов имеет уникальные особенности, преимущества и недостатки.

Традиционный метод автоматического распознавания речи представляет собой совокупность компьютерного оборудования и программных технологий, выполняющих прямую идентификацию и обработку человеческого голоса. Принцип работы данной технологии может быть определена в качестве автоматической транскрипции разговорного языка в читаемый текст. При этом распознавание голоса происходит в режиме реального времени на основе заранее заданных звуковых шаблонов. Таким образом, данный метод ASR представляет компьютеру возможность выявить слова из человеческой речи и перевести их в электронный текст~\cite{Khlopenkova}.

\subsection{Обзор предметной области программного продукта}

Проект по разработке голосового помощника с использованием технологии автоматического распознавания речи (ASR) направлен на создание интеллектуальной системы, способной интерпретировать и реагировать на человеческую речь в режиме реального времени. С развитием технологий ASR голосовые помощники стали неотъемлемой частью улучшения пользовательского опыта в самых разных сферах — от умного дома до клиентского сервиса. Цель данного проекта — исследовать, спроектировать и оценить голосового помощника, который сможет точно распознавать устную речь, обрабатывать различные акценты и контексты, а также эффективно выполнять команды пользователя. В рамках данного исследования рассматриваются методы повышения точности преобразования речи в текст, снижения задержек и обеспечения надежного распознавания намерений, что в конечном итоге позволит создать голосовой интерфейс, ощущающийся естественным, быстрым и удобным для пользователя.

Голосовой помощник, использующий технологию автоматического распознавания речи (ASR), представляет собой мощное сочетание технологий обработки речи и естественного языка, обеспечивающее естественное и удобное голосовое взаимодействие между человеком и машиной. Такие системы позволяют пользователям управлять устройствами, выполнять задачи и получать информацию с помощью голосовых команд, исключая необходимость физического ввода и делая технологии более доступными. Технология ASR преобразует устную речь в текст, который система может интерпретировать, а обработка естественного языка (NLP) помогает распознавать намерения пользователя и формировать соответствующие ответы или действия. Благодаря широкому спектру применения — от умных домов до виртуальной поддержки клиентов — голосовые помощники на базе ASR становятся важнейшим инструментом для обеспечения более плавного, персонализированного и эффективного взаимодействия с цифровыми системами. Данный проект посвящен разработке и оптимизации такого голосового помощника с акцентом на повышение точности преобразования речи в текст, распознавания намерений и скорости отклика, чтобы создать максимально интуитивный пользовательский опыт.


Обзор литературы для проекта под названием ``Анализ эмоциональных тонов в разговорах с виртуальными помощниками'' включает изучение исследований, связанных с распознаванием эмоций в системах разговорного искусственного интеллекта, в частности, в виртуальных ассистентах. В рамках этого обзора рассматриваются работы по анализу эмоционального тона, методы машинного обучения для распознавания эмоций и практические применения виртуальных помощников. Обзор предоставляет информацию о методах и подходах, используемых для обнаружения эмоций в речи, включая методы обработки аудиосигналов, модели машинного обучения и соответствующие наборы данных.


 Рассматриваются различные методы обработки естественного языка (NLP) для выявления эмоций в тексте, включая анализ тональности, использование эмоциональных лексиконов и модели машинного обучения. Оцениваются достоинства и недостатки этих подходов, подчёркивается важность эффективного извлечения признаков, контекстного понимания и семантической интеграции. Обзор акцентирует внимание на влиянии распознавания эмоций на сферы обслуживания клиентов и психического здоровья, а также подчёркивает необходимость дальнейших исследований для повышения точности и адаптируемости методов в разных языках и контекстах.
	
 Представлен обзор методов анализа голосовых характеристик, таких как высота тона и тембр, для обнаружения эмоций. Рассматриваются как традиционные статистические и машинно-обучающие методы, так и современные достижения в области глубокого обучения. Основными вызовами остаются вариативность эмоциональных выражений и культурные различия. Обзор обсуждает практическое применение этих методов в сферах обслуживания клиентов, психического здоровья и взаимодействия человек-компьютер, а также определяет направления для будущих исследований, направленных на повышение точности и надёжности систем распознавания эмоций.
	
 Обзор методов обнаружения эмоций в взаимодействиях человек-робот с использованием речи, мимики и физиологических сигналов. Отмечаются достижения в области машинного обучения и искусственного интеллекта, которые способствовали повышению точности и обработке данных в режиме реального времени. Рассматриваются проблемы, связанные с разнообразием эмоциональных выражений и чувствительностью к контексту. Обзор подчёркивает необходимость разработки надёжных систем для улучшения эмоционального взаимодействия в таких приложениях, как социальная робототехника и здравоохранение.
	
 Представлена концепция гибридной модели, объединяющей сверточные нейронные сети (CNN) и сети долгосрочной краткосрочной памяти (LSTM) для повышения точности распознавания эмоций в речи. Такой подход позволяет эффективно учитывать как пространственные, так и временные характеристики аудиосигналов, превосходя традиционные методы. В исследовании отмечены значительные улучшения точности на различных наборах данных эмоций, что делает данный метод перспективным для применения в системах распознавания эмоций в реальном времени, таких как взаимодействие человек-компьютер и аффективные вычисления.
	
 Рассматривается использование технологий глубокого обучения для повышения эффективности распознавания эмоциональной речи во взаимодействиях человек-компьютер. Предложена новая архитектура нейронной сети, интегрирующая различные подходы глубокого обучения для повышения точности и персонализации распознавания эмоций по аудиосигналам. Эффективность предложенной модели подтверждена обширными экспериментами, показавшими значительные улучшения в распознавании нюансированных эмоциональных состояний и соответствующей адаптации откликов системы. Данное направление открывает возможности для создания более отзывчивых и эмоционально интеллектуальных систем в таких областях, как виртуальные помощники и технологии клиентского сервиса.
 
 С ростом спроса на бесшовное и удобное взаимодействие с технологиями без использования рук традиционные интерфейсы, такие как сенсорный ввод или набор текста, всё чаще сталкиваются с ограничениями. Это особенно заметно в ситуациях, когда критически важны доступность, удобство или высокая скорость взаимодействия. Существующие голосовые помощники, несмотря на их широкое применение, по-прежнему испытывают трудности с точностью распознавания речи в условиях языкового многообразия, при наличии различных акцентов, фонового шума, а также при интерпретации контекстно-зависимых команд~\cite{name22}.

Разрабатываемое программное обеспечение представляет собой голосовой помощник, функционирующий на одноплатном компьютере Raspberry Pi 4.
Основная цель проекта — создание стабильной и ресурсоэффективной системы, способной работать в автономном режиме с голосовым управлением.
Программа должна эффективно использовать вычислительные ресурсы Raspberry Pi 4.

Аппаратные требования:
\begin{itemize}
	\item Максимальную загрузку процессора не более 40\% при обработке голосовых команд;
	\item Потребление оперативной памяти в пределах 300 МБ для предотвращения избыточной нагрузки на систему.
\end{itemize}

Голосовая активация и автономная работа:
\begin{itemize}
	\item Функционал активируется только по ключевому слову "Алиса" после чего система переходит в режим ожидания команды;
	\item Обработка речи и выполнение команд должны работать без интернета, обеспечивая полную автономность.
\end{itemize}

Запуск и интерфейс:
\begin{itemize}
	\item Приложение должно запускаться из командной строки, без необходимости графического интерфейса.
\end{itemize}

Эти требования направлены на создание надежного и энергоэффективного решения, способного стабильно работать на Raspberry Pi 4 с ограниченными ресурсами.
\newpage
