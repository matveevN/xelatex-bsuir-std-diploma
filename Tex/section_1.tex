\section{Анализ литературных источников и формирование требований к проектируемому программному обеспечению}

\subsection{Аналитический обзор литературных источников}

XXI век по праву можно охарактеризовать как эпоху информационного взрыва, ключевыми признаками которого стали стремительное развитие науки, технологий и цифровых систем. Одной из наиболее значимых тенденций современного этапа научно-технического прогресса является возрастающий интерес к интеллектуальным формам взаимодействия человека и компьютера. Эта тенденция отражает глобальную цифровую трансформацию общества и находит практическое воплощение в разработке и использовании новых средств коммуникации, среди которых особое место занимает голосовое управление. Наиболее перспективным направлением в данной области выступают автоматизированные системы распознавания речи, обеспечивающие возможность интуитивного и естественного взаимодействия пользователя с машиной посредством голосовых команд.

Формирование и становление речевых технологий началось ещё в конце XX века, когда были заложены теоретические основы автоматической обработки речи. Особую роль в этом процессе сыграли научные труды таких исследователей, как Б.М. Лобанов, Т.К. Винцюк, А.В. Фролов, Л.Р. Рабинер, Р.В. Шафер, У.А. Ли, Д.Х. Клетт, X.D. Huang, H.W. Hon, A. Acero и др. Ими были предложены и реализованы первые эффективные модели автоматического распознавания речи, направленные на решение как фундаментальных, так и прикладных задач: от создания алгоритмов выделения речевых признаков до разработки практических приложений для трансформации речи в текст. Значительное число научных публикаций отечественных и зарубежных специалистов в этой области подтверждает важность и актуальность темы. Более того, даже спустя десятилетия интерес к проблемам речевых технологий не ослабевает, а напротив -- усиливается на фоне всеобщей цифровизации и стремления к созданию человекоцентричных интерфейсов взаимодействия~\cite{LL1}.

На современном этапе развития информационных технологий системы распознавания речи становятся неотъемлемой частью повседневной жизни. Они активно внедряются в потребительскую электронику, мобильные устройства, автомобильные системы, корпоративные сервисы, а также в сферы образования, здравоохранения, безопасности и управления. Голосовое взаимодействие с компьютером становится более интуитивным и эффективным по сравнению с традиционными способами ввода информации, такими как клавиатура и мышь. Возможность управления устройствами при помощи голоса делает технологии более доступными для широкой аудитории, включая пожилых людей и пользователей с ограниченными возможностями. В этом контексте речевые технологии не просто дополняют традиционные интерфейсы, а формируют новые стандарты пользовательского опыта~\cite{LL2}.

Кроме того, автоматизированные системы распознавания и синтеза речи способны выполнять целый ряд функций, ранее доступных исключительно человеку. Среди них -- автоматическое озвучивание текстовых материалов, преобразование устной речи в письменную форму, выявление ключевых слов и фрагментов в аудиофайлах большого объема, что значительно снижает затраты времени на их обработку. Такие возможности особенно актуальны в условиях информационной перегрузки, характерной для современной эпохи.

Важным направлением практического применения речевых технологий является создание интеллектуальных голосовых помощников и цифровых агентов. Эти системы активно используются в смартфонах, умных колонках, автомобилях, банках, колл-центрах и других областях. К числу распространённых примеров можно отнести такие сервисы, как Siri, Alexa, Google Assistant, «Алиса» от Яндекса и др.~\cite{Hein}. Они выполняют функции персональных ассистентов: принимают команды, отвечают на вопросы, управляют устройствами «умного дома», помогают организовывать расписание и выполнять повседневные задачи, тем самым значительно повышая удобство жизни современного пользователя.

Особый интерес представляет использование распознавания речи в образовательной сфере. Современные обучающие платформы на основе речевых интерфейсов позволяют реализовать автоматическое проведение экзаменов, голосовую проверку знаний, обучение произношению и пониманию иностранной речи. Такие технологии способствуют персонализации образовательного процесса и обеспечивают интерактивную форму взаимодействия между студентом и цифровой системой. Это особенно актуально в условиях дистанционного и гибридного обучения, где голосовой интерфейс может выполнять функцию основного канала коммуникации.

Более того, речевые технологии находят применение в специализированных областях профессиональной деятельности. К таким сферам относятся биометрическая идентификация, криминалистика, судебная экспертиза, здравоохранение, телемедицина, автоматизация служб поддержки и помощь людям с ограниченными возможностями. Это расширяет спектр задач, которые могут быть решены при помощи интеллектуальных речевых систем, и свидетельствует о высоком потенциале этих технологий для дальнейшего развития~\cite{Muravyov}.

В основе современных систем распознавания речи лежат различные методологические подходы. На сегодняшний день выделяют два основных направления: ASR и инновационный комплексный end-to-end подход~\cite{Riqiang}. Оба метода находят широкое применение и имеют свои особенности, преимущества и недостатки.

Традиционный метод ASR предполагает раздельную обработку речевого сигнала поэтапно: сначала производится выделение акустических признаков и преобразование их в фонемы, затем в слова и, наконец, в предложения. Такая архитектура позволяет достичь модульности и универсальности, поскольку отдельные компоненты могут быть адаптированы под различные языки, акценты и условия окружающей среды. Однако данный подход сопряжён с определёнными ограничениями: потеря информации на каждом из этапов преобразования, сложность настройки и адаптации к новой предметной области, ограниченная гибкость при работе с нерегламентированной речью~\cite{Khlopenkova}.

В противоположность ему, комплексный end-to-end подход предлагает полностью сквозную архитектуру, основанную на использовании глубоких нейросетевых моделей. Речь преобразуется в текст напрямую, без промежуточного выделения фонем или языковых моделей. Такой подход обеспечивает более высокую точность и адаптивность в условиях динамически изменяющейся речевой среды. Примером работы может служить следующая последовательность: захват аудиосигнала, его преобразование в текст, анализ смысла и контекста, уточнение гипотезы и формирование команды к исполнению. Эта модель приближает процесс взаимодействия к естественному человеческому общению и всё чаще используется в современных голосовых помощниках~\cite{Abougarair}.

Основные характеристики, отличия, области применения, преимущества и недостатки данных методов сведены в таблицу~\ref{tab:traditional-vs-complex-sr}, что позволяет наглядно сравнить подходы и выбрать наиболее подходящий под конкретную задачу.

\begin{table}[H]
	\caption{Сравнение методов традиционного и комплексного распознания речи}
	\label{tab:traditional-vs-complex-sr}
	\centering 
	\begin{tblr}{
			width=\textwidth,
			colspec={X[1]|X[2]|X[2]},
			vlines,
		}
		\hline 
		  & Традиционное распознание речи  & End-to-end распознание речи \\ 
		\hline  
		Преимущества &
			-- модульность архитектуры ПО на основе этой архитектуры \newline
			-- повторное использование одних и тех же решений для схожих языков \newline
			-- высокая производительность, работа в реальном времени
		  & 
		  	-- отсутствие потери данных из-за преобразования данных в различные
		  	промежуточные абстракции \newline
		  -- возможность учета контекста на уровне
		  распознания речи \newline
		  -- высокое качество распознания \\
		\hline  
		Недостатки  &
			-- нет возможности учета контекста в ходе распознания \newline
			-- низкое качество распознания сленговых выражений, имен собственных, аббревиатур
		& 
			-- более высокие требования к производительности \newline
			-- высокие требования к качеству и размеру корпуса текстов для обучения
		 \\ 
		\hline  
		Особенности   &
		
	     Сочетание высокой производительности
	     и модульности позволяет использовать
	     даже на портативных устройствах.
		
		& Высокое качество распознания при высоких
		требованиях к производительности на
		сегодняшний день ограничивают
		применение облачными технологиями. \\ 
		\hline  
		Области
		использования   &
		Широко применяется в случаях
		необходимости распознания речи в
		реальном времени или распознания на
		локальных устройствах – голосовые
		помощники, «умные» устройства с
		голосовым управлением.
		
		& Является перспективным решением для
		извлечения текстовых стенограмм,
		системах корпоративного и
		государственного управления,
		перспективных голосовых помощниках с
		обработкой голоса в облаке. \\ 
		\hline
	\end{tblr}
	
\end{table}


Благодаря сравнительному анализу, приведённому в таблице~\ref{tab:traditional-vs-complex-sr}, можно сделать вывод, что оба подхода -- как традиционный, так и комплексный end-to-end -- имеют свои особенности и применимы в различных условиях. При этом современные голосовые помощники всё чаще ориентируются на гибридные или end-to-end архитектуры, поскольку они обеспечивают более высокую точность и лучшее понимание контекста.

Такие системы постепенно выходят за рамки простого распознавания текста, расширяя свои возможности за счёт анализа интонации, эмоциональной окраски и контекстуальных особенностей речи. Это делает взаимодействие с голосовыми интерфейсами более естественным и приближённым к человеческому общению.

Таким образом, развитие технологий распознавания речи представляет собой важный этап на пути к созданию по-настоящему интеллектуальных и человеко-ориентированных цифровых систем. Интеграция голосового управления в повседневную жизнь способствует формированию удобной, адаптивной и эффективной среды взаимодействия между человеком и машиной.

\subsection{Обзор предметной области программного продукта}

Современные информационные технологии стремительно развиваются в направлении создания более естественных форм взаимодействия человека с компьютером. Одним из наиболее значимых достижений в этой области является ASR лежащее в основе голосовых помощников -- интеллектуальных систем, способных воспринимать и интерпретировать устную речь пользователя в реальном времени.

Голосовые помощники, построенные на базе ASR, совмещают технологии обработки аудиосигналов и NLP, что позволяет им распознавать команды, интерпретировать смысл высказываний и выполнять различные действия без необходимости физического ввода. Такие системы находят широкое применение в умных домах, мобильных устройствах, автомобильных интерфейсах, а также в службах поддержки клиентов. За счёт голосового управления повышается доступность цифровых сервисов, особенно для людей с ограниченными возможностями, а взаимодействие становится более быстрым и интуитивно понятным.

Целью настоящего проекта является разработка голосового помощника, способного надёжно распознавать устную речь, адаптироваться к различным акцентам, шумовым условиям и контексту высказывания, а также эффективно интерпретировать намерения пользователя. Особое внимание уделяется таким аспектам, как повышение точности преобразования речи в текст, снижение задержек и улучшение механизмов понимания смысла сказанного. Это необходимо для создания голосового интерфейса, обеспечивающего естественное и комфортное взаимодействие.

В рамках проекта также рассматривается задача анализа эмоционального состояния пользователя на основе речевых данных. Интеграция механизмов распознавания эмоций расширяет функциональность голосового помощника, позволяя не только фиксировать команды, но и учитывать эмоциональный контекст общения. Это особенно важно в сферах клиентского сервиса, дистанционного образования и цифрового здравоохранения, где эмоциональная отзывчивость системы играет существенную роль.

Современные исследования в области анализа эмоционального тона речи фокусируются на использовании методов машинного и глубокого обучения, таких как сверточные нейронные сети, сети с долговременной краткосрочной памятью, а также гибридных архитектур. Для извлечения признаков эмоций из речи анализируются акустические параметры (высота, интенсивность, тембр), а также контекстуальные особенности текста. Большое внимание уделяется межкультурным различиям в выражении эмоций, обработке многоязычных данных и адаптации моделей к индивидуальным особенностям пользователей.

Несмотря на достижения в данной области, голосовые помощники продолжают сталкиваться с рядом вызовов. Наиболее критичными остаются точность распознавания речи в условиях шума, корректная интерпретация неоднозначных команд, а также устойчивость моделей к разнообразию акцентов и эмоциональных состояний. Именно эти аспекты становятся предметом анализа и оптимизации в рамках настоящего проекта.
На рисунке \ref{fig:VoiceAssistant} представлена схема  взаимодействия пользователя с голосовым помощником.
 \begin{figure}[H]
 	\centering
 	\fbox{\includegraphics[scale=0.7]{VoiceAssistant.jpg}}
 	\caption{Схема взаимодействия пользователя с голосовым помощником}
 	\label{fig:VoiceAssistant}
 \end{figure}
 
 
 Данная схема показывает взаимодействие между пользователем и голосовым помощником. Пользователь отправляет запрос голосовому помощнику. Помощник обрабатывает запрос, формирует ответ и отправляет его обратно пользователю. Это взаимодействие может повторяться многократно, позволяя пользователю вести диалог с голосовым помощником.
 
 Ответ помощника может быть представлен в виде текста, речи или других форм вывода. Стрелки между пользователем и помощником указывают на поток информации: пользователь отправляет запросы, а помощник возвращает ответы.
 
 \begin{itemize}
 	\item Пользователь: Начинает взаимодействие, отправляя запрос.
 	\item Голосовой помощник: Обрабатывает запрос и формирует ответ.
 	\item Запрос: Ввод или вопрос пользователя к голосовому помощнику.
 	\item Ответ: Результат или решение, предоставляемое помощником в ответ на запрос.
 \end{itemize}
 
 Эта простая модель взаимодействия отражает базовый принцип общения между пользователем и голосовым помощником, что необходимо для естественного и эффективного взаимодействия~\cite{name22}.
 
Разрабатываемое программное обеспечение представляет собой голосовой помощник, функционирующий на одноплатном компьютере Raspberry Pi 4 Model B.
Основная цель проекта -- создание стабильной и ресурсоэффективной системы, способной работать в автономном режиме с голосовым управлением.
Программа должна эффективно использовать вычислительные ресурсы Raspberry Pi 4 Model B.

Аппаратные требования:
\begin{itemize}
	\item Максимальную загрузку процессора не более 20\% при обработке голосовых команд;
	\item Потребление оперативной памяти в пределах 300 МБ для предотвращения избыточной нагрузки на систему.
\end{itemize}

Голосовая активация и автономная работа:
\begin{itemize}
	\item Функционал активируется только по ключевому слову "Алиса" после чего система переходит в режим ожидания команды;
	\item Обработка речи и выполнение команд должны работать без интернета, обеспечивая полную автономность.
\end{itemize}

Запуск и интерфейс:
\begin{itemize}
	\item Приложение должно запускаться из командной строки, без необходимости графического интерфейса.
\end{itemize}

Эти требования направлены на создание надежного и энергоэффективного решения, способного стабильно работать на Raspberry Pi 4 Model B с ограниченными ресурсами.
\newpage
