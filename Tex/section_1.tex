\section{Анализ литературных источников и формирование требований к проектируемому программному обеспечению}

\subsection{Аналитический обзор литературных источников}

XXI век можно с уверенностью назвать эпохой «информационного взрыва», отличительной чертой которого является процесс активного развития науки и техники. Следует обратить внимание на одну из важнейших особенностей современного этапа технического прогресса, которая заключается
в повышении интереса к изучению интеллектуальных форм
взаимодействия человека и компьютера, одним из наиболее
перспективных на данный момент видов которого можно
считать взаимодействие человека и компьютера при помощи голосовых команд. В связи с этим наиболее актуальным
видом интеллектуальных систем являются автоматизированные системы распознавания речи.

Конец ХХ и начало XXI в. ознаменовались началом истории развития речевых технологий, важную роль в которых получили системы распознавания речи. Наиболее известными в этой области являются работы следующих авторов:
Б.М. Лобанова, Т.К. Винцюка, А.В. Фролова, Л.Р. Рабине-
ра, Р.В. Шафера, У.А. Ли, Д.Х. Клетта, X.D. Huang, H.-W. Hon,
A. Acero. Тот период был значимым в плане решения большого числа как фундаментальных, так и прикладных задач
в сфере обработки, анализа и синтеза речевых сигналов.
Об этом свидетельствует большое число трудов зарубежных
и российских исследователей. посвященных данному вопросу. Однако и в наше время интерес к проблеме обработки
речевых сигналов не угасает: напротив, она остается одной
из наиболее актуальных и активно развивающихся в настоящее время~\cite{LL1}.


Многие исследователи говорят и пишут о том, что технологии распознавания речи все более прочно входят в нашу жизнь, легко находя себе применение в самых разных сферах человеческой жизни, делая ее намного проще. Речевые технологии предоставляют возможность общения компьютера и человека посредством речи без использования каких-либо иных способов ввода информации, таких, как, например, клавиатура, безусловно, это более удобно для людей. Использование речевых технологий также помогает переложить часть функций операторов на компьютер: так, с помощью автоматизированных систем синтеза и анализа речи представляется возможным читать книги, сообщения, озвучить документы, а также целые веб-сайты. Благодаря речевым системам можно создавать интеллектуальные системы обучения, которые могут принимать несложные экзамены и даже помогать в изучении иностранных языков. С помощью систем распознавания речи становится возможным улавливать в потоке речи
заданные звуковые фрагменты, так называемые ключевые
слова. Особенно это важно, если аудио файл достаточно
большого объема: на обработку такого файла оператором
ушло бы очень большое количество времени, тогда как с по-
мощью автоматизированных систем распознавания речи
можно значительно ускорить этот процесс~\cite{LL2}.


В современном мире наблюдаются устойчивые тенденции, связанные с развитием и интеграцией в различных бытовых и профессиональных сферах жизнедеятельности технологий распознавания голоса. На основе данных инструментов представляется возможность взаимодействия между человеком и цифровым устройством исключительно на основе использования голоса. Использование методов распознавания голоса является основой работы современных голосовых помощников, способных упростить и качественно улучшить жизнь современного человека. Наиболее распространёнными из примеров использования данных технологий являются индивидуальные голосовые помощники, телефонные роботы и ряд иных задач, при выполнении которых происходит экономия временных ресурсов человека~\cite{Hein}.

Таким образом, голосовое управление становится не просто удобным дополнением, но и важным элементом цифровой среды, формируя новые стандарты взаимодействия между человеком и техникой.

Помимо этого, в зависимости от используемых методов распознавания голоса, активно разрабатываются интеллектуальные системы обучения, способные принимать экзамен и помогать в изучении иностранных языков.
Это открывает принципиально новые возможности для персонализации образовательного процесса, особенно в дистанционном обучении, где голосовой интерфейс может стать основным каналом взаимодействия.
При этом использование голосовых помощников получает свое развитие и в решении наиболее сложных задач из профессиональных сфер человеческой жизнедеятельности. Примерами данного применения являются – идентификация личности, судебная экспертиза, помощь людям с ограниченными возможностями и ряд иных задач~\cite{Muravyov}.

Применение технологий распознавания речи в таких сферах демонстрирует их высокий потенциал для решения задач, ранее доступных исключительно человеку, что делает данные разработки особенно значимыми в условиях стремительной цифровизации.

Широкое распространение голосовых помощников стало возможным благодаря развитию методов распознавания голоса. На сегодняшний день существует два основных направления развития по распознаванию речи – традиционное направление автоматического распознавания речи и направление комплексного (end-to-end) подхода к распознаванию речи~\cite{Riqiang}. При этом необходимо отметить, что каждый из данных методов имеет уникальные особенности, преимущества и недостатки.

Традиционный метод автоматического распознавания речи представляет собой совокупность компьютерного оборудования и программных технологий, выполняющих прямую идентификацию и обработку человеческого голоса. Принцип работы данной технологии может быть определена в качестве автоматической транскрипции разговорного языка в читаемый текст. При этом распознавание голоса происходит в режиме реального времени на основе заранее заданных звуковых шаблонов. Таким образом, данный метод ASR представляет компьютеру возможность выявить слова из человеческой речи и перевести их в электронный текст~\cite{Khlopenkova}.

Основное использование данного метода наблюдается в задачах распознавания слов для идентификации речи человека. Данный метод подразумевает разделение ответственности за распознание различных компонентов языка – последовательное преобразование звука в фонемы, фонем в слова, слов – в связные предложения. Подобная архитектура определяет с одной стороны преимущества данного подхода – модульность и возможность повторного использования некоторых данных компонентов для разных языков, имеющих похожие фонемы, словоформы или синтаксис.
Несмотря на свою эффективность и проверенность временем, данный подход может уступать в гибкости и адаптивности более современным методам, особенно в условиях динамически меняющейся речевой среды.
В то же время, из этого выходят и недостатки подобного подхода – потеря ряда данных, на каждом из преобразований.

Особое внимание заслуживает инновационный метод комплексного подхода к распознаванию речи. Данный метод представляет собой актуальное направление из области машинного обучения, в котором обработка голоса производится на основе интеллектуальных алгоритмов. Данный подход отличается от традиционного тем, что фонемы преобразуются напрямую в текст, минуя промежуточные формы представления. При этом один из вариантов процесса распознавания голоса в данном методе, примененного в голосовом помощнике, может выглядеть таким образом: запись речи человека; преобразование машиной слов из аудио в электронный текст; разбор текста на основные составляющие для понимания контекста беседы и целей человека; После определения контекста, алгоритм может возвратиться к этапу анализа аудио для более точного преобразования в текст и повторить цикл; по результатам работы система определяет команду на выполнение~\cite{Abougarair}.

В таблице~\ref{tab:traditional-vs-complex-sr} сведены основные особенности, сферы использования, преимущества и недостатки рассмотренных методов распознавания голоса.

\begin{table}[H]
	\caption{Сравнение методов традиционного и комплексного распознания речи}
	\label{tab:traditional-vs-complex-sr}
	\centering 
	\begin{tblr}{
			width=\textwidth,
			colspec={X[1]|X[2]|X[2]},
			vlines,
		}
		\hline 
		  & Традиционное распознание речи  & End-to-end распознание речи \\ 
		\hline  
		Преимущества &
			-- модульность архитектуры ПО на основе этой архитектуры \newline
			-- повторное использование одних и тех же решений для схожих языков \newline
			-- высокая производительность, работа в реальном времени
		  & 
		  	-- отсутствие потери данных из-за преобразования данных в различные
		  	промежуточные абстракции \newline
		  -- возможность учета контекста на уровне
		  распознания речи \newline
		  -- высокое качество распознания \\
		\hline  
		Недостатки  &
			-- нет возможности учета контекста в ходе распознания \newline
			-- низкое качество распознания сленговых выражений, имен собственных, аббревиатур
		& 
			-- более высокие требования к производительности \newline
			-- высокие требования к качеству и размеру корпуса текстов для обучения
		 \\ 
		\hline  
		Особенности   &
		
	     Сочетание высокой производительности
	     и модульности позволяет использовать
	     даже на портативных устройствах.
		
		& Высокое качество распознания при высоких
		требованиях к производительности на
		сегодняшний день ограничивают
		применение облачными технологиями. \\ 
		\hline  
		Области
		использования   &
		Широко применяется в случаях
		необходимости распознания речи в
		реальном времени или распознания на
		локальных устройствах – голосовые
		помощники, «умные» устройства с
		голосовым управлением.
		
		& Является перспективным решением для
		извлечения текстовых стенограмм,
		системах корпоративного и
		государственного управления,
		перспективных голосовых помощниках с
		обработкой голоса в облаке. \\ 
		\hline
	\end{tblr}
	
\end{table}


Благодаря такому подходу, современные голосовые помощники становятся всё более "умными", способными не только распознать речь, но и учитывать интонацию, эмоциональную окраску и контекст, что значительно приближает их к естественному человеческому общению.

В целом, развитие технологий распознавания речи можно рассматривать как важный шаг на пути к созданию по-настоящему интуитивных и человеко-ориентированных цифровых систем, способных адаптироваться к потребностям пользователя и контексту задачи.

\subsection{Обзор предметной области программного продукта}

Проект по разработке голосового помощника с использованием технологии автоматического распознавания речи (ASR) направлен на создание интеллектуальной системы, способной интерпретировать и реагировать на человеческую речь в режиме реального времени. С развитием технологий ASR голосовые помощники стали неотъемлемой частью улучшения пользовательского опыта в самых разных сферах — от умного дома до клиентского сервиса. Цель данного проекта — исследовать, спроектировать и оценить голосового помощника, который сможет точно распознавать устную речь, обрабатывать различные акценты и контексты, а также эффективно выполнять команды пользователя. В рамках данного исследования рассматриваются методы повышения точности преобразования речи в текст, снижения задержек и обеспечения надежного распознавания намерений, что в конечном итоге позволит создать голосовой интерфейс, ощущающийся естественным, быстрым и удобным для пользователя.

Голосовой помощник, использующий технологию автоматического распознавания речи (ASR), представляет собой мощное сочетание технологий обработки речи и естественного языка, обеспечивающее естественное и удобное голосовое взаимодействие между человеком и машиной. Такие системы позволяют пользователям управлять устройствами, выполнять задачи и получать информацию с помощью голосовых команд, исключая необходимость физического ввода и делая технологии более доступными. Технология ASR преобразует устную речь в текст, который система может интерпретировать, а обработка естественного языка (NLP) помогает распознавать намерения пользователя и формировать соответствующие ответы или действия. Благодаря широкому спектру применения — от умных домов до виртуальной поддержки клиентов — голосовые помощники на базе ASR становятся важнейшим инструментом для обеспечения более плавного, персонализированного и эффективного взаимодействия с цифровыми системами. Данный проект посвящен разработке и оптимизации такого голосового помощника с акцентом на повышение точности преобразования речи в текст, распознавания намерений и скорости отклика, чтобы создать максимально интуитивный пользовательский опыт.


Обзор литературы для проекта под названием "Анализ эмоциональных тонов в разговорах с виртуальными помощниками" включает изучение исследований, связанных с распознаванием эмоций в системах разговорного искусственного интеллекта, в частности, в виртуальных ассистентах. В рамках этого обзора рассматриваются работы по анализу эмоционального тона, методы машинного обучения для распознавания эмоций и практические применения виртуальных помощников. Обзор предоставляет информацию о методах и подходах, используемых для обнаружения эмоций в речи, включая методы обработки аудиосигналов, модели машинного обучения и соответствующие наборы данных.


 Рассматриваются различные методы обработки естественного языка (NLP) для выявления эмоций в тексте, включая анализ тональности, использование эмоциональных лексиконов и модели машинного обучения. Оцениваются достоинства и недостатки этих подходов, подчёркивается важность эффективного извлечения признаков, контекстного понимания и семантической интеграции. Обзор акцентирует внимание на влиянии распознавания эмоций на сферы обслуживания клиентов и психического здоровья, а также подчёркивает необходимость дальнейших исследований для повышения точности и адаптируемости методов в разных языках и контекстах.
	
 Представлен обзор методов анализа голосовых характеристик, таких как высота тона и тембр, для обнаружения эмоций. Рассматриваются как традиционные статистические и машинно-обучающие методы, так и современные достижения в области глубокого обучения. Основными вызовами остаются вариативность эмоциональных выражений и культурные различия. Обзор обсуждает практическое применение этих методов в сферах обслуживания клиентов, психического здоровья и взаимодействия человек-компьютер, а также определяет направления для будущих исследований, направленных на повышение точности и надёжности систем распознавания эмоций.
	
 Обзор методов обнаружения эмоций в взаимодействиях человек-робот с использованием речи, мимики и физиологических сигналов. Отмечаются достижения в области машинного обучения и искусственного интеллекта, которые способствовали повышению точности и обработке данных в режиме реального времени. Рассматриваются проблемы, связанные с разнообразием эмоциональных выражений и чувствительностью к контексту. Обзор подчёркивает необходимость разработки надёжных систем для улучшения эмоционального взаимодействия в таких приложениях, как социальная робототехника и здравоохранение.
	
 Представлена концепция гибридной модели, объединяющей сверточные нейронные сети (CNN) и сети долгосрочной краткосрочной памяти (LSTM) для повышения точности распознавания эмоций в речи. Такой подход позволяет эффективно учитывать как пространственные, так и временные характеристики аудиосигналов, превосходя традиционные методы. В исследовании отмечены значительные улучшения точности на различных наборах данных эмоций, что делает данный метод перспективным для применения в системах распознавания эмоций в реальном времени, таких как взаимодействие человек-компьютер и аффективные вычисления.
	
 Рассматривается использование технологий глубокого обучения для повышения эффективности распознавания эмоциональной речи во взаимодействиях человек-компьютер. Предложена новая архитектура нейронной сети, интегрирующая различные подходы глубокого обучения для повышения точности и персонализации распознавания эмоций по аудиосигналам. Эффективность предложенной модели подтверждена обширными экспериментами, показавшими значительные улучшения в распознавании нюансированных эмоциональных состояний и соответствующей адаптации откликов системы. Данное направление открывает возможности для создания более отзывчивых и эмоционально интеллектуальных систем в таких областях, как виртуальные помощники и технологии клиентского сервиса.
 
 С ростом спроса на бесшовное и удобное взаимодействие с технологиями без использования рук традиционные интерфейсы, такие как сенсорный ввод или набор текста, всё чаще сталкиваются с ограничениями. Это особенно заметно в ситуациях, когда критически важны доступность, удобство или высокая скорость взаимодействия. Существующие голосовые помощники, несмотря на их широкое применение, по-прежнему испытывают трудности с точностью распознавания речи в условиях языкового многообразия, при наличии различных акцентов, фонового шума, а также при интерпретации контекстно-зависимых команд.
 
 \begin{figure}[H]
 	\centering
 	\includegraphics[scale=0.7]{VoiceAssistant.jpg}
 	\caption{Схема взаимодействия пользователя с голосовым помощником}
 	\label{fig:VoiceAssistant}
 \end{figure}
 
 
 Диаграмма также показывает взаимодействие между пользователем и виртуальным ассистентом. Пользователь отправляет запрос (вопрос или команду) виртуальному ассистенту. Ассистент обрабатывает запрос, формирует ответ и отправляет его обратно пользователю. Это взаимодействие может повторяться многократно, позволяя пользователю вести диалог с виртуальным ассистентом.
 
 Ответ ассистента может быть представлен в виде текста, речи или других форм вывода. Стрелки между пользователем и ассистентом указывают на поток информации: пользователь отправляет запросы, а ассистент возвращает ответы.
 
 \begin{itemize}
 	\item Пользователь: Начинает взаимодействие, отправляя запрос.
 	\item Виртуальный помощник: Обрабатывает запрос и формирует ответ.
 	\item Запрос: Ввод или вопрос пользователя к виртуальному помощнику.
 	\item Ответ: Результат или решение, предоставляемое помощником в ответ на запрос.
 \end{itemize}
 
 Эта простая модель взаимодействия отражает базовый принцип общения между пользователем и виртуальным помощником, что необходимо для естественного и эффективного взаимодействия~\cite{name22}.
 
 
 

Разрабатываемое программное обеспечение представляет собой голосовой помощник, функционирующий на одноплатном компьютере Raspberry Pi 4 Model B.
Основная цель проекта — создание стабильной и ресурсоэффективной системы, способной работать в автономном режиме с голосовым управлением.
Программа должна эффективно использовать вычислительные ресурсы Raspberry Pi 4 Model B.

Аппаратные требования:
\begin{itemize}
	\item Максимальную загрузку процессора не более 20\% при обработке голосовых команд;
	\item Потребление оперативной памяти в пределах 300 МБ для предотвращения избыточной нагрузки на систему.
\end{itemize}

Голосовая активация и автономная работа:
\begin{itemize}
	\item Функционал активируется только по ключевому слову "Алиса" после чего система переходит в режим ожидания команды;
	\item Обработка речи и выполнение команд должны работать без интернета, обеспечивая полную автономность.
\end{itemize}

Запуск и интерфейс:
\begin{itemize}
	\item Приложение должно запускаться из командной строки, без необходимости графического интерфейса.
\end{itemize}

Эти требования направлены на создание надежного и энергоэффективного решения, способного стабильно работать на Raspberry Pi 4 Model B с ограниченными ресурсами.
\newpage
