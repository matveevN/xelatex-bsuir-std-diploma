\section{Обзор и анализ способов управления одноплатным компьютером и формировани требований к проектируемому программному обеспечению}

\subsection{Аналитический обзор литературных источников}

В современном мире наблюдаются устойчивые тенденции, связанные с развитием и интеграцией в различных бытовых и профессиональных сферах жизнедеятельности технологий распознавания голоса. На основе данных инструментов представляется возможность взаимодействия между человеком и цифровым устройством исключительно на основе использования голоса. Использование методов распознавания голоса является основой работы современных голосовых помощников, способных упростить и качественно улучшить жизнь современного человека. Наиболее распространёнными из примеров использования данных технологий являются индивидуальные голосовые помощники, телефонные роботы и ряд иных задач, при выполнении которых
происходит экономия временных ресурсов человека~\cite{Hein}.

\subsection{Обзор предметной области программного продукта}

\subsection{Требования к проектируемому программному обеспечению}

Разрабатываемое программное обеспечение представляет собой голосовой ассистент, функционирующий на одноплатном компьютере Raspberry Pi 4.
Основная цель проекта — создание стабильной и ресурсоэффективной системы, способной работать в автономном режиме с голосовым управлением.
Программа должна эффективно использовать вычислительные ресурсы Raspberry Pi 4.

Аппаратные требования:
\begin{itemize}
	\item Максимальную загрузку процессора не более 40\% при обработке голосовых команд;
	\item Потребление оперативной памяти в пределах 300 МБ для предотвращения избыточной нагрузки на систему.
\end{itemize}

Голосовая активация и автономная работа:
\begin{itemize}
	\item Функционал активируется только по ключевому слову "Алиса" после чего система переходит в режим ожидания команды;
	\item Обработка речи и выполнение команд должны работать без интернета, обеспечивая полную автономность.
\end{itemize}

Запуск и интерфейс:
\begin{itemize}
	\item Приложение должно запускаться из командной строки, без необходимости графического интерфейса.
\end{itemize}

Эти требования направлены на создание надежного и энергоэффективного решения, способного стабильно работать на Raspberry Pi 4 с ограниченными ресурсами.
\newpage
