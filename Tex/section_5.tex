\section{Технико-экономическое обоснование }

\subsection{Характеристика разработанного по индивидуальному заказу
	программного средства}

Разработанное в дипломном проекте программное средство для голосового управления промышленным оборудованием с использованием технологий Vosk, С++ и одноплатного компьютера предназначено для использования заказчиком с целью автоматизации управления станками и производственными линиями без использования рук.

Основные функции программного продукта:

\begin{itemize}
	\item голосовое управление оборудованием;
	
	\item оффлайн-работа без зависимости от интернета;
	
	\item гибкость конфигурации;
	
	\item повышение безопасности.

\end{itemize}

Основными заказчиками и пользователями являются:

\begin{itemize}
	\item малые и средние производственные предприятия;
	
	\item операторы станков;
	
	\item инженеры, нуждающиеся в голосовом управлении без отрыва от работы.
	
\end{itemize}


Применение программного продукта в производственно-хозяйственной деятельности обеспечит заказчику:
\begin{itemize}
	\item снижение времени простоя за счет быстрого голосового управления;
	\item уменьшение количества ошибок при вводе команд;
	\item повышение безопасности за счет отсутствия необходимости касаться панелей управления грязными руками;
	\item снижение затрат на автоматизацию по сравнению с промышленными SCADA-системами.
\end{itemize}


Основной потребностью в использовании программного продукта заказчиком является упрощение управления оборудованием и повышение эффективности производства. 

Аналогичные решения (например, Siemens Voice Control) требуют дорогостоящей интеграции, а облачные системы (Alexa, Google Assistant):
\begin{itemize}
	\item не работают оффлайн;
	\item не адаптированы для промышленных условий.
\end{itemize}


\subsection{Расчет затрат на разработку и цена программного средства, созданного по индивидуальному заказу} 
\label{subsec:dev_cost_calculation}

Для разработки программного средства для одноплатного компьютера необходимы следующие специалисты: программист на С++ и специалист по тестированию программного обеспечения. Трудозатраты на разработку программного средства у программиста на С++ будут равны 140 часам, в то время как у специалиста по тестированию программного обеспечения они будут равны 50 часам.

Расчет основной заработной платы разработчиков осуществляется по формуле \ref{eq:base_salary}:
\begin{equation}
	\label{eq:base_salary}
	\mathrm{З_{о}} =  \mathrm{К_{пр}} \sum_{i=1}^n \text{З}_{\text{ч}i} \cdot t_i,
\end{equation}

где $n$ - количество человек, задействованных в разработке голосового помощника; 
$\mathrm{К_{пр}}$ - коэффициент премий; 
$\text{З}_{\text{ч}i}$ - часовая заработная плата $i$-го исполнителя (руб.); 
$t_i$ - трудоёмкость работ, выполняемых $i$-ым исполнителем.

Примем количество рабочих часов в месяце равным 168 часам. Расчет затрат на основную заработную плату разработчикам приведен в таблице~\ref{tab:salary}. 


\begin{table}[H]
	\caption{Расчет затрат на основную заработную плату команды разработчиков}
	\label{tab:salary}
	\centering 
	\begin{tblr}{
			width=\textwidth,
			colspec={*{5}{X[c]}},  % Все столбцы центрированы по умолчанию
			cell{4-6}{1-4} = {l},  % Объединенные ячейки (строки 4-6, столбцы 1-4) - по левому краю
			vlines,
		}
		\hline 
		Категория исполнителя & Месячный оклад, р. & Часовой оклад, р. & Трудоемкость работ, ч & Итого, р. \\ 
		\hline  
		Программист
		& 2400
		& 14,3
		& 140
		& 2002   \\
		\hline  
		Тестировщик & 1800  & 10,7  & 50  & 535  \\ 
		\hline   
		\SetCell[r=1,c=4]{11.4cm}{Итого}& & & & 2537 \\ 
		\hline   
		\SetCell[r=1,c=4]{11.4cm}{Премия и иные стимулирующие выплаты (50 \%)} & & & & 1268,5 \\ 
		\hline  
		\SetCell[r=1,c=4]{11.4cm}{Всего затрат на основную заработную плату разработчиков} & & & & 3805,5 \\ 
		
		\hline 
		
	\end{tblr}
	
\end{table}

Формирование цены программного средства на основе затрат приведено в таблице~\ref{tab:price-calculation}. 

\begin{table}[H]
	\caption{Формирования цены программного средства на основе затрат}
	\label{tab:price-calculation}
	\centering 
	\begin{tblr}{
			width=\textwidth,
			colspec={X[2,c,m]|X[3,c,m]|X[c,m]},
			vlines,
		}
		\hline 
		Наименование статьи затрат  & Формула/таблица для расчёта & Значение, р. \\ 
		\hline  
		Основная заработная плата разработчиков & Таблица~\ref{tab:salary}  & 3805,5 \\
		\hline  
		Дополнительная заработная плата разработчиков  &
		
		\neqt{\text{З}_\text{д}=\frac{ \text{З}_\text{о}\cdot\text{Н}_\text{д}}{100}}
		где	$\text{Н}_\text{д}$ – норматив дополнительной заработной платы (10 \%)
		\eqt{{\text{З}}_{\text{д}} = \frac{3805{,}5 \cdot 10}{100}}
		
		
		& 380,55 \\ 
		\hline  
		Отчисления на социальные нужды   &
		
		\neqt{\text{Р}_\text{соц}=\frac{ (\text{З}_\text{о}+\text{З}_\text{д})\cdot\text{Н}_\text{соц}}{100}}
		где	$\text{Н}_\text{соц}$ – норматив отчислений от фонда оплаты труда (34,6 \%)
		\eqt{{\text{Р}}_{\text{соц}} = \frac{(3805{,}5 + 380,55) \cdot 34{,}6}{100}}
		
		
		& 1448,37 \\ 
		\hline  
		Прочие расходы    &
		
		\neqt{\text{Р}_\text{пр}=\frac{ \text{З}_\text{о}\cdot\text{Н}_\text{пр}}{100}}
		где	$\text{Н}_\text{пр}$ –  норматив прочих расходов (30 \%)
		\eqt{{\text{Р}}_{\text{пр}} = \frac{3805{,}5\cdot 30}{100}}
		
		
		& 1141,65 \\ 
		\hline
	\end{tblr}
	
\end{table}
\newpage

\noindent Продолжение таблицы \ref{tab:price-calculation}
\begin{center}
	\begin{tblr}{
			width=\textwidth,
			colspec={X[2,c,m]|X[3,c,m]|X[c,m]},
			vlines,
		}
		\hline 
		Наименование статьи затрат  & Формула/таблица для расчёта & Значение, р. \\   
		\hline
		Общая сумма затрат на разработку    &
		
		\neqt{\label{eq:total_cost}\text{З}_\text{р} = \text{З}_\text{о} + \text{З}_\text{д} + \text{Р}_\text{соц} + \text{Р}_\text{пр}}
		\eqt{
			\begin{aligned}
				\text{З}_\text{р} &= 3805{,}5 + 380{,}55 + 1448{,}37 \\ &\quad + 1141{,}65
			\end{aligned}
		}
		
		& 6776,07 \\ 
		\hline 
		Плановая прибыль, включаемая в
		цену программного средства  & 
		
		\neqt{\mathrm{П_{п.c}} = \frac{\mathrm{З_{р}} \cdot \mathrm{Р_{п.с}}}{100}}			
		где $\text{Р}_\text{п.с}$ – рентабельность затрат на разработку программного средства (25\%)
		
		\eqt{\mathrm{П_{п.c}} = \frac{6776{,}07 \cdot 25}{100}}
		
		& 1694,01   \\
		\hline
		Отпускная цена программного
		средства & 
		\neqt{\mathrm{Ц_{п.c}} = {\mathrm{З_{р}} + \mathrm{П_{п.с}}}}
		\eqt{\mathrm{Ц_{п.c}} = {6776{,}07} + {1694{,}01}}
		& 8470,08  \\
		\hline		
	\end{tblr}
\end{center}



В результате расчетов, приведенных в таблице \ref{tab:price-calculation}, была получена отпускная цена программного средства, равная 8470,08 рублям. Цена получилась выше, чем предлагается фриланс-разработчиками в интернете по результатам поиска Google, и в то же время на порядок меньше, чем у более серьезных фирм-разработчиков. 


\subsection{Расчет результата от разработки и использования программного средства, созданного по индивидуальному заказу} 
\label{subsec:development_results}     

Для организации-разработчика экономическим эффектом является прирост чистой прибыли, полученной от разработки и реализации программного средства заказчику.
Прибыль программного средства, реализованного организацией-разработчиком по отпускной цене, сформированной на основе затрат на разработку, рассчитывается по следующей формуле \ref{eq:clean_profit}:
\begin{equation}
	\label{eq:clean_profit}   
	\Delta 	\mathrm{П_{ч}} = \mathrm{П_{п.с}} \left( 1 - \frac{	\mathrm{Н_{п}}}{100} \right),
\end{equation}

где $\text{П}_{\text{п.с}}$ – прибыль, включаемая в цену программного средства, $\text{Н}_{\text{п}}$ – налог на прибыль. Так как компания является резидентом Парка высоких технологий, то оно освобождается от налога на прибыль. Поэтому получим соответствующее значение:
\[
\Delta \mathrm{П_{ч}} = 1694{,}01 \left(1 - \frac{0}{100}\right) = 1694{,}01 \text{ р}.
\]

Для организации-заказчика экономическим эффектом является снижение затрат или увеличение прибыли, полученное за счет использования программного средства, разработанного по индивидуальному заказу. Экономия на заработной плате и начислениях на заработную плату сотрудников за счет снижения трудоемкости работ рассчитывается по формуле \ref{eq:savings_salary}:
\begin{equation}
	\label{eq:savings_salary}
	\mathrm{Э_{з.п}} = \mathrm{К_{пр}} \cdot (t_p^{\text{без п.с}} - t_p^{\text{с п.с}}) \cdot \mathrm{Т_{ч}} \cdot N_{\text{п}} \cdot \left(1 + \frac{\mathrm{Н_{д}}}{100}\right) \cdot \left(1 + \frac{\mathrm{Н_{соц}}}{100}\right)
\end{equation}

где $\mathrm{К_{пр}}$ – коэффициент премий равен 2; $t_\text{р}^{\text{без п.с}}$, $t_\text{р}^{\text{с п.с}}$ ‒ трудоемкость выполнения работ сотрудниками до (0,25 ч) и после внедрения программного средства (0,033 ч); $\text{Т}_{\text{ч}}$ ‒ часовой оклад сотрудника, использующего программное средство, 12,5 р.; $N_{\text{п}}$ – плановый объем работ, выполняемых сотрудником равен 1000; $\mathrm{Н_{д}}$ – норматив дополнительной заработной платы (10 \% ); $\mathrm{Н_{соц}}$ – ставка отчислений от заработной платы, включаемых в себестоимость (34,6 \%). Подставив соответствующие значения в формулу \ref{eq:savings_salary}, получим:
\[
\mathrm{Э_{з.п}} = 2 \cdot (0{,}25 - 0{,}033 ) \cdot 12{,}5 \cdot 1000 \cdot \left(1 + \frac{10}{100}\right) \cdot \left(1 + \frac{34{,}6}{100}\right) = 8032{,}26  \text{ р}.
\]

Экономическим эффектом при использовании программного средства является прирост чистой прибыли, полученной за счет экономии на текущих затратах предприятия, который рассчитывается по формуле \ref{eq:net_profit_increase}:
\begin{equation}
	\label{eq:net_profit_increase}
	\Delta \mathrm{П_{ч}} = (\mathrm{Э_{з.п}} - \Delta \text{З}_\text{тек}^{\text{п.с}}) \left( 1 - \frac{\mathrm{Н_{п}}}{100} \right),
\end{equation}

где $\mathrm{Э_{тек}}$ – экономия на текущих затратах при использовании программного средства, р.; $\Delta \text{З}_\text{тек}^{\text{п.с}}$ – прирост текущих затрат, связанных с использованием программного средства р.; $\mathrm{Н_{п}}$ ‒ ставка налога на прибыль согласно действующему законодательству. Подставив соответствующие значения в формулу \ref{eq:net_profit_increase}, получим:
\[
\Delta \mathrm{П_{ч}} = (8032{,}26 - 100) \left(1 - \frac{0}{100}\right) = 7932{,}26 \text{ р}.
\]

\subsection{Расчет показателей экономической эффективности разработки и использования программного средства} 

Для организации-разработчика программного средства оценка экономической эффективности разработки осуществляется с помощью расчета простой нормы прибыли (рентабельности инвестиций (затрат) на разработку программного средства) по формуле \ref{eq:roi}:
\begin{equation}
	\label{eq:roi}
	\mathrm{Р_{и}} = \frac{\Delta \mathrm{П_{ч}}}{\mathrm{З_{р}}} \cdot 100\%,
\end{equation}

Подставив значения, посчитанные в разделах \ref{subsec:dev_cost_calculation} и \ref{subsec:development_results}   по формулам \ref{eq:total_cost} и \ref{eq:clean_profit}, в формулу, получим следующий результат:
\[
\mathrm{Р_{и}} = \frac{1694{,}01}{6776{,}07} \cdot 100\% = 25\%.
\]

Для расчёта показателей экономической эффективности разработки и использования приложения необходимо полученные суммы результата чистой прибыли и затрат инвестиций в разработку программного средства по годам привести к единому моменту времени – расчётному году (2025 г.) путём умножения результатов и затрат за каждый год на коэффициент дисконтирования($\alpha_t$) который рассчитывается по формуле:
\begin{equation}
	\alpha_t = \frac{1}{(1 + d)^{t - t_p}},
\end{equation}
где d – требуемая норма дисконта, 15\%;
t – номер года, результаты и затраты которого приводятся к расчётному (2025 – 1, 2026 – 2, 2027 – 3).
\[
\alpha_1 = \frac{1}{(1+0{,}15)^0} = 1,
\]
\[
\alpha_2 = \frac{1}{(1+0{,}15)^1} = 0{,}87,
\]
\[
\alpha_3 = \frac{1}{(1+0{,}15)^2} = 0{,}76
\]

Расчёт показателей экономической эффективности разработки и реализации программного средства приведено в таблице \ref{tab:efficiency_metrics}. 

\begin{table}[H]
	\caption{Расчёт показателей экономической эффективности разработки и реализации программного средства}
	\label{tab:efficiency_metrics}
	\centering 
	\begin{tblr}{
			width=\textwidth,
			colspec={X[4,l]|X[1.5,c,m]|X[1.5,c,m]|X[1.5,c,m]|X[c,m]},
			cell{3-11}{1} = {l},  % Объединенные ячейки (строки 3-11, столбец 1) - по левому краю
			vlines,
		}
		\hline 
		\SetCell[r=2]{c} Показатель & \SetCell[c=3]{c} Значение по годам расчётного периода
		& &  \\ 
		\hline  
		& 2025 & 2026 & 2027  \\
		\hline    
		\textbf{Результат} &  &  & \\
		\hline  
		1 Прирост чистой прибыли от реализации, р.  & 7932,26 &  7932,26 &  7932,26 \\ 
		\hline  
		2 Дисконтированный результат, р. & 7932,26 & 6900,07 & 6028,96 \\ 
		\hline  
		\textbf{Затраты} & & &  \\ 
		\hline
		3 Инвестиции в разработку программного средства, р. & 8470,08 & -- & -- \\
		\hline
		4 Дисконтированные инвестиции, р. & 8470,08 & -- & -- \\
		\hline
		5 Чистый дисконтированный доход по годам, р. & -537,82 & 6900,07 & 6028,96  \\
		\hline
		6 Чистый дисконтированный доход нарастающим итогом, р & -1965,63 & 6362,25 & 12391,21 \\
		\hline
		7 Коэффициент дисконтирования, доли единицы & 1 & 0,87 & 0,76 \\
		\hline
	\end{tblr}
\end{table}


Простой срок окупаемости инвестиций определяется по формуле \ref{eq:payback_period}:
\begin{equation}
	\label{eq:payback_period}
	\mathrm{Т_{ок}} = \frac{\sum_{t=1}^{n} \text{З}_t}{\frac{1}{n} \cdot \sum_{t=1}^{n} \Delta \text{П}_{\text{ч}t}},
\end{equation}

где $n$ – расчетный период, лет;  $\text{З}_t$ – затраты (инвестиции)
в году $t$, р.*;  $\Delta\text{П}_{\text{ч}t}$ – прирост чистой прибыли в году $t$ в результате реализации проекта, р.
Подставив соответствующие значения в формулу \ref{eq:payback_period}, получим:

\[
\text{Т}_{\text{ок}} = \frac{8470{,}08}{\frac{1}{3} \cdot (7932{,}26 + 6900{,}07 + 6028{,}96)} = 1{,}22 
\]

Данный показатель говорит о том, что уже на 2 год эксплуатации программного продукта, заказчик окупит вложенные в закупку программного продукта денежные средства.

Расчёт средней нормы прибыли (рентабельности инвестиций) производится по формуле \ref{eq:arr}: 
\begin{equation}
	\label{eq:arr}
	\mathrm{Р_{и}} = \frac{\frac{1}{n} \cdot \sum_{t=1}^n \Delta \Pi_{\text{ч}t}}{\sum_{t=1}^n \text{З}_t} \cdot 100\%
\end{equation}

Подставив соответствующие значения в формулу \ref{eq:arr}, получим:
\[
\mathrm{Р_{и}}  = \frac{\frac{1}{3} \cdot \left(7932{,}26 + 6900{,}07 + 6028{,}96\right)}{8470{,}08} \cdot 100\% = 82{,}10\%
\]

Полученная рентабельность говорит о том, что организации-заказчику будет выгодно вложить свои средства в внедрение разрабатываемое программное средство.

В результате технико-экономического обоснования разработки и использования голосового помощника для одноплатного компьютера были получены результаты, которые свидетельствуют об эффективности разработки:

\begin{itemize}
	\item общая сумма расходов составила 8470,08 рубля;
	\item уровень рентабельности составил 82,10\%.
\end{itemize}

Таким образом, использование программного средства является экономически эффективным и инвестиции в его разработку целесообразно осуществлять.


\newpage
